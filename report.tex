\documentclass[a4paper,11pt]{article}

\usepackage[a4paper,bindingoffset=0.2in,left=0.8in,right=0.8in,top=0.5in,bottom=1in,footskip=.25in]{geometry}
\usepackage{amsmath}
\usepackage{amsfonts}
\usepackage{amssymb}
\usepackage{txfonts}
\usepackage{fontspec}
\usepackage{newpxtext}

\usepackage{agda}
\usepackage{newunicodechar}
\newunicodechar{→}{\ensuremath{\rightarrow}}
\newunicodechar{←}{\ensuremath{\leftarrow}}
\newunicodechar{×}{\ensuremath{\times}}
\newunicodechar{∀}{\ensuremath{\forall}}
\newunicodechar{≡}{\ensuremath{\equiv}}
\newunicodechar{≅}{\ensuremath{\cong}}
\newunicodechar{≐}{\ensuremath{\doteq}}
\newunicodechar{∈}{\ensuremath{\in}}
\newunicodechar{∧}{\ensuremath{\land}}
\newunicodechar{∨}{\ensuremath{\lor}}
\newunicodechar{⊤}{\ensuremath{\top}}
\newunicodechar{⊥}{\ensuremath{\bot}}
\newunicodechar{⊔}{\ensuremath{\sqcup}}
\newunicodechar{∷}{\ensuremath{{::}}}
\newunicodechar{ℓ}{\ensuremath{\ell}}
\newunicodechar{₀}{\ensuremath{{_\text{0}}}}
\newunicodechar{₁}{\ensuremath{{_\text{1}}}}
\newunicodechar{₂}{\ensuremath{{_\text{2}}}}
\newunicodechar{₃}{\ensuremath{{_\text{3}}}}
\newunicodechar{₄}{\ensuremath{{_\text{4}}}}
\newunicodechar{₅}{\ensuremath{{_\text{5}}}}
\newunicodechar{₆}{\ensuremath{{_\text{6}}}}
\newunicodechar{₇}{\ensuremath{{_\text{7}}}}
\newunicodechar{₈}{\ensuremath{{_\text{8}}}}
\newunicodechar{₉}{\ensuremath{{_\text{9}}}}
\newunicodechar{⟨}{\ensuremath{{\langle}}}
\newunicodechar{⟩}{\ensuremath{{\rangle}}}
\newunicodechar{∘}{\ensuremath{\circ}}
\newunicodechar{ℕ}{\ensuremath{\mathbb{N}}}
\newunicodechar{λ}{\ensuremath{\lambda}}
\newunicodechar{Π}{\ensuremath{\Pi}}
\newunicodechar{Σ}{\ensuremath{\Sigma}}
\newunicodechar{Γ}{\ensuremath{\Gamma}}
\newunicodechar{Δ}{\ensuremath{\Delta}}
\newunicodechar{ρ}{\ensuremath{\rho}}
\newunicodechar{α}{\ensuremath{\alpha}}
\newunicodechar{ƛ}{\ensuremath{\lambdabar}}
\newunicodechar{σ}{\ensuremath{\sigma}}
\newunicodechar{β}{\ensuremath{\beta}}
\newunicodechar{ξ}{\ensuremath{\xi}}
\newunicodechar{γ}{\ensuremath{\gamma}}
\newunicodechar{ε}{\ensuremath{\varepsilon}}
\newunicodechar{ι}{\ensuremath{\iota}}
\newunicodechar{ₕ}{\ensuremath{{_h}}}
\newunicodechar{∋}{\ensuremath{\ni}}
\newunicodechar{⇒}{\ensuremath{\Rightarrow}}
\newunicodechar{⟶}{\ensuremath{\longrightarrow}}
\newunicodechar{∅}{\ensuremath{\emptyset}}
\newunicodechar{≤}{\ensuremath{\leq}}
\newunicodechar{∙}{\ensuremath{\bullet}}
\newunicodechar{∎}{\ensuremath{\blacksquare}}
\newunicodechar{μ}{\ensuremath{\mu}}

\title{{\large{Final Project}}\\Soundness of grab/delimit, send/run, and Effect Handlers}
\date{\today}
\author{Satoshi Takimoto}

\begin{document}

\maketitle

In this report, we are going to prove using Agda the soundness property of the lambda calculus with each of grab/delimit, send/run, and effect handlers.
This report is written in Literate Agda. The full source code, tested with agda 2.6.4 and agda-stdlib 2.0, is available on GitHub\footnote[1]{URL}.

\section{Introduction}

Based on PLFA\cite{plfa22.08}. Intrincically typed, De Bruijn representation.

\section{grab/delimit}

We begin with grab/delimit.

\begin{code}[hide]%
\>[0]\AgdaKeyword{module}\AgdaSpace{}%
\AgdaModule{GrabDelimit}\AgdaSpace{}%
\AgdaKeyword{where}\<%
\\
%
\\[\AgdaEmptyExtraSkip]%
\>[0]\AgdaKeyword{open}\AgdaSpace{}%
\AgdaKeyword{import}\AgdaSpace{}%
\AgdaModule{Data.Empty}\AgdaSpace{}%
\AgdaKeyword{using}\AgdaSpace{}%
\AgdaSymbol{(}\AgdaSpace{}%
\AgdaFunction{⊥}\AgdaSymbol{;}\AgdaSpace{}%
\AgdaFunction{⊥-elim}\AgdaSpace{}%
\AgdaSymbol{)}\<%
\\
\>[0]\AgdaKeyword{open}\AgdaSpace{}%
\AgdaKeyword{import}\AgdaSpace{}%
\AgdaModule{Relation.Binary.PropositionalEquality}\AgdaSpace{}%
\AgdaKeyword{using}\AgdaSpace{}%
\AgdaSymbol{(}\AgdaSpace{}%
\AgdaOperator{\AgdaDatatype{\AgdaUnderscore{}≡\AgdaUnderscore{}}}\AgdaSymbol{;}\AgdaSpace{}%
\AgdaInductiveConstructor{refl}\AgdaSymbol{;}\AgdaSpace{}%
\AgdaFunction{cong}\AgdaSpace{}%
\AgdaSymbol{)}\<%
\\
\>[0]\AgdaKeyword{open}\AgdaSpace{}%
\AgdaKeyword{import}\AgdaSpace{}%
\AgdaModule{Data.Nat}\AgdaSpace{}%
\AgdaKeyword{using}\AgdaSpace{}%
\AgdaSymbol{(}\AgdaSpace{}%
\AgdaDatatype{ℕ}\AgdaSymbol{;}\AgdaSpace{}%
\AgdaInductiveConstructor{zero}\AgdaSymbol{;}\AgdaSpace{}%
\AgdaInductiveConstructor{suc}\AgdaSymbol{;}\AgdaSpace{}%
\AgdaOperator{\AgdaFunction{\AgdaUnderscore{}<\AgdaUnderscore{}}}\AgdaSymbol{;}\AgdaSpace{}%
\AgdaOperator{\AgdaFunction{\AgdaUnderscore{}≤?\AgdaUnderscore{}}}\AgdaSymbol{;}\AgdaSpace{}%
\AgdaInductiveConstructor{z≤n}\AgdaSymbol{;}\AgdaSpace{}%
\AgdaInductiveConstructor{s≤s}\AgdaSpace{}%
\AgdaSymbol{)}\<%
\\
\>[0]\AgdaKeyword{open}\AgdaSpace{}%
\AgdaKeyword{import}\AgdaSpace{}%
\AgdaModule{Relation.Nullary}\AgdaSpace{}%
\AgdaKeyword{using}\AgdaSpace{}%
\AgdaSymbol{(}\AgdaSpace{}%
\AgdaOperator{\AgdaFunction{¬\AgdaUnderscore{}}}\AgdaSpace{}%
\AgdaSymbol{)}\<%
\\
\>[0]\AgdaKeyword{open}\AgdaSpace{}%
\AgdaKeyword{import}\AgdaSpace{}%
\AgdaModule{Relation.Nullary.Decidable}\AgdaSpace{}%
\AgdaKeyword{using}\AgdaSpace{}%
\AgdaSymbol{(}\AgdaSpace{}%
\AgdaFunction{True}\AgdaSymbol{;}\AgdaSpace{}%
\AgdaFunction{toWitness}\AgdaSpace{}%
\AgdaSymbol{)}\<%
\end{code}

\begin{code}[hide]%
\>[0]\AgdaKeyword{infixr}\AgdaSpace{}%
\AgdaNumber{8}\AgdaSpace{}%
\AgdaOperator{\AgdaFunction{↑\AgdaUnderscore{}}}\<%
\\
\>[0]\AgdaKeyword{infixr}\AgdaSpace{}%
\AgdaNumber{7}\AgdaSpace{}%
\AgdaOperator{\AgdaInductiveConstructor{\AgdaUnderscore{}⇒\AgdaUnderscore{}!\AgdaUnderscore{}}}\<%
\\
\>[0]\AgdaKeyword{infixl}\AgdaSpace{}%
\AgdaNumber{7}\AgdaSpace{}%
\AgdaOperator{\AgdaInductiveConstructor{\AgdaUnderscore{}·\AgdaUnderscore{}}}\<%
\\
\>[0]\AgdaKeyword{infixl}\AgdaSpace{}%
\AgdaNumber{5}\AgdaSpace{}%
\AgdaOperator{\AgdaInductiveConstructor{\AgdaUnderscore{},\AgdaUnderscore{}}}\<%
\\
\>[0]\AgdaKeyword{infix}%
\>[7]\AgdaNumber{5}\AgdaSpace{}%
\AgdaOperator{\AgdaInductiveConstructor{ƛ\AgdaUnderscore{}}}\AgdaSpace{}%
\AgdaOperator{\AgdaInductiveConstructor{μ\AgdaUnderscore{}}}\<%
\\
\>[0]\AgdaKeyword{infixl}\AgdaSpace{}%
\AgdaNumber{6}\AgdaSpace{}%
\AgdaOperator{\AgdaFunction{\AgdaUnderscore{}[\AgdaUnderscore{}]}}\AgdaSpace{}%
\AgdaOperator{\AgdaFunction{\AgdaUnderscore{}⟨\AgdaUnderscore{}⟩}}\<%
\\
\>[0]\AgdaKeyword{infix}%
\>[7]\AgdaNumber{4}\AgdaSpace{}%
\AgdaOperator{\AgdaDatatype{\AgdaUnderscore{}∋\AgdaUnderscore{}}}\<%
\\
\>[0]\AgdaKeyword{infix}%
\>[7]\AgdaNumber{2}\AgdaSpace{}%
\AgdaOperator{\AgdaDatatype{\AgdaUnderscore{}⟶\AgdaUnderscore{}}}\AgdaSpace{}%
\AgdaOperator{\AgdaFunction{\AgdaUnderscore{}⟶*\AgdaUnderscore{}}}\<%
\end{code}

\subsection{Syntax}

\begin{code}%
\>[0]\AgdaKeyword{data}\AgdaSpace{}%
\AgdaDatatype{Ty}\AgdaSpace{}%
\AgdaSymbol{:}\AgdaSpace{}%
\AgdaPrimitive{Set}\<%
\\
\>[0]\AgdaKeyword{data}\AgdaSpace{}%
\AgdaDatatype{Eff}\AgdaSpace{}%
\AgdaSymbol{:}\AgdaSpace{}%
\AgdaPrimitive{Set}\<%
\\
%
\\[\AgdaEmptyExtraSkip]%
\>[0]\AgdaKeyword{data}\AgdaSpace{}%
\AgdaDatatype{Ty}\AgdaSpace{}%
\AgdaKeyword{where}\<%
\\
\>[0][@{}l@{\AgdaIndent{0}}]%
\>[2]\AgdaInductiveConstructor{`ℕ}\AgdaSpace{}%
\AgdaSymbol{:}\AgdaSpace{}%
\AgdaDatatype{Ty}\<%
\\
%
\>[2]\AgdaOperator{\AgdaInductiveConstructor{\AgdaUnderscore{}⇒\AgdaUnderscore{}!\AgdaUnderscore{}}}\AgdaSpace{}%
\AgdaSymbol{:}\AgdaSpace{}%
\AgdaDatatype{Ty}\AgdaSpace{}%
\AgdaSymbol{→}\AgdaSpace{}%
\AgdaDatatype{Ty}\AgdaSpace{}%
\AgdaSymbol{→}\AgdaSpace{}%
\AgdaDatatype{Eff}\AgdaSpace{}%
\AgdaSymbol{→}\AgdaSpace{}%
\AgdaDatatype{Ty}\<%
\\
%
\\[\AgdaEmptyExtraSkip]%
\>[0]\AgdaKeyword{data}\AgdaSpace{}%
\AgdaDatatype{Eff}\AgdaSpace{}%
\AgdaKeyword{where}\<%
\\
\>[0][@{}l@{\AgdaIndent{0}}]%
\>[2]\AgdaInductiveConstructor{ι}\AgdaSpace{}%
\AgdaSymbol{:}\AgdaSpace{}%
\AgdaDatatype{Eff}\<%
\\
%
\>[2]\AgdaOperator{\AgdaInductiveConstructor{\AgdaUnderscore{},\AgdaUnderscore{}}}\AgdaSpace{}%
\AgdaSymbol{:}\AgdaSpace{}%
\AgdaDatatype{Eff}\AgdaSpace{}%
\AgdaSymbol{→}\AgdaSpace{}%
\AgdaDatatype{Ty}\AgdaSpace{}%
\AgdaSymbol{→}\AgdaSpace{}%
\AgdaDatatype{Eff}\<%
\\
%
\\[\AgdaEmptyExtraSkip]%
\>[0]\AgdaKeyword{data}\AgdaSpace{}%
\AgdaDatatype{Ctx}\AgdaSpace{}%
\AgdaSymbol{:}\AgdaSpace{}%
\AgdaPrimitive{Set}\AgdaSpace{}%
\AgdaKeyword{where}\<%
\\
\>[0][@{}l@{\AgdaIndent{0}}]%
\>[2]\AgdaInductiveConstructor{∙}\AgdaSpace{}%
\AgdaSymbol{:}\AgdaSpace{}%
\AgdaDatatype{Ctx}\<%
\\
%
\>[2]\AgdaOperator{\AgdaInductiveConstructor{\AgdaUnderscore{},\AgdaUnderscore{}}}\AgdaSpace{}%
\AgdaSymbol{:}\AgdaSpace{}%
\AgdaDatatype{Ctx}\AgdaSpace{}%
\AgdaSymbol{→}\AgdaSpace{}%
\AgdaDatatype{Ty}\AgdaSpace{}%
\AgdaSymbol{→}\AgdaSpace{}%
\AgdaDatatype{Ctx}\<%
\\
%
\\[\AgdaEmptyExtraSkip]%
\>[0]\AgdaKeyword{variable}\<%
\\
\>[0][@{}l@{\AgdaIndent{0}}]%
\>[2]\AgdaGeneralizable{Γ}\AgdaSpace{}%
\AgdaGeneralizable{Δ}\AgdaSpace{}%
\AgdaSymbol{:}\AgdaSpace{}%
\AgdaDatatype{Ctx}\<%
\\
%
\>[2]\AgdaGeneralizable{α}\AgdaSpace{}%
\AgdaGeneralizable{β}\AgdaSpace{}%
\AgdaGeneralizable{γ}\AgdaSpace{}%
\AgdaGeneralizable{αₕ}\AgdaSpace{}%
\AgdaGeneralizable{βₕ}\AgdaSpace{}%
\AgdaSymbol{:}\AgdaSpace{}%
\AgdaDatatype{Ty}\<%
\\
%
\>[2]\AgdaGeneralizable{ε}\AgdaSpace{}%
\AgdaGeneralizable{ε'}\AgdaSpace{}%
\AgdaGeneralizable{ε''}\AgdaSpace{}%
\AgdaSymbol{:}\AgdaSpace{}%
\AgdaDatatype{Eff}\<%
\\
%
\\[\AgdaEmptyExtraSkip]%
\>[0]\AgdaKeyword{data}\AgdaSpace{}%
\AgdaOperator{\AgdaDatatype{\AgdaUnderscore{}∋\AgdaUnderscore{}}}\AgdaSpace{}%
\AgdaSymbol{:}\AgdaSpace{}%
\AgdaDatatype{Ctx}\AgdaSpace{}%
\AgdaSymbol{→}\AgdaSpace{}%
\AgdaDatatype{Ty}\AgdaSpace{}%
\AgdaSymbol{→}\AgdaSpace{}%
\AgdaPrimitive{Set}\AgdaSpace{}%
\AgdaKeyword{where}\<%
\\
\>[0][@{}l@{\AgdaIndent{0}}]%
\>[2]\AgdaInductiveConstructor{zero}\AgdaSpace{}%
\AgdaSymbol{:}\AgdaSpace{}%
\AgdaGeneralizable{Γ}\AgdaSpace{}%
\AgdaOperator{\AgdaInductiveConstructor{,}}\AgdaSpace{}%
\AgdaGeneralizable{α}\AgdaSpace{}%
\AgdaOperator{\AgdaDatatype{∋}}\AgdaSpace{}%
\AgdaGeneralizable{α}\<%
\\
%
\>[2]\AgdaInductiveConstructor{suc}\AgdaSpace{}%
\AgdaSymbol{:}\AgdaSpace{}%
\AgdaGeneralizable{Γ}\AgdaSpace{}%
\AgdaOperator{\AgdaDatatype{∋}}\AgdaSpace{}%
\AgdaGeneralizable{α}\AgdaSpace{}%
\AgdaSymbol{→}\AgdaSpace{}%
\AgdaGeneralizable{Γ}\AgdaSpace{}%
\AgdaOperator{\AgdaInductiveConstructor{,}}\AgdaSpace{}%
\AgdaGeneralizable{β}\AgdaSpace{}%
\AgdaOperator{\AgdaDatatype{∋}}\AgdaSpace{}%
\AgdaGeneralizable{α}\<%
\\
%
\\[\AgdaEmptyExtraSkip]%
\>[0]\AgdaKeyword{data}\AgdaSpace{}%
\AgdaDatatype{Tm}\AgdaSpace{}%
\AgdaSymbol{:}\AgdaSpace{}%
\AgdaDatatype{Ctx}\AgdaSpace{}%
\AgdaSymbol{→}\AgdaSpace{}%
\AgdaDatatype{Eff}\AgdaSpace{}%
\AgdaSymbol{→}\AgdaSpace{}%
\AgdaDatatype{Ty}\AgdaSpace{}%
\AgdaSymbol{→}\AgdaSpace{}%
\AgdaPrimitive{Set}\AgdaSpace{}%
\AgdaKeyword{where}\<%
\\
\>[0][@{}l@{\AgdaIndent{0}}]%
\>[2]\AgdaInductiveConstructor{var}\AgdaSpace{}%
\AgdaSymbol{:}\AgdaSpace{}%
\AgdaGeneralizable{Γ}\AgdaSpace{}%
\AgdaOperator{\AgdaDatatype{∋}}\AgdaSpace{}%
\AgdaGeneralizable{α}\AgdaSpace{}%
\AgdaSymbol{→}\AgdaSpace{}%
\AgdaDatatype{Tm}\AgdaSpace{}%
\AgdaGeneralizable{Γ}\AgdaSpace{}%
\AgdaGeneralizable{ε}\AgdaSpace{}%
\AgdaGeneralizable{α}\<%
\\
%
\>[2]\AgdaOperator{\AgdaInductiveConstructor{ƛ\AgdaUnderscore{}}}\AgdaSpace{}%
\AgdaSymbol{:}\AgdaSpace{}%
\AgdaDatatype{Tm}\AgdaSpace{}%
\AgdaSymbol{(}\AgdaGeneralizable{Γ}\AgdaSpace{}%
\AgdaOperator{\AgdaInductiveConstructor{,}}\AgdaSpace{}%
\AgdaGeneralizable{α}\AgdaSymbol{)}\AgdaSpace{}%
\AgdaGeneralizable{ε}\AgdaSpace{}%
\AgdaGeneralizable{β}\AgdaSpace{}%
\AgdaSymbol{→}\AgdaSpace{}%
\AgdaDatatype{Tm}\AgdaSpace{}%
\AgdaGeneralizable{Γ}\AgdaSpace{}%
\AgdaGeneralizable{ε'}\AgdaSpace{}%
\AgdaSymbol{(}\AgdaGeneralizable{α}\AgdaSpace{}%
\AgdaOperator{\AgdaInductiveConstructor{⇒}}\AgdaSpace{}%
\AgdaGeneralizable{β}\AgdaSpace{}%
\AgdaOperator{\AgdaInductiveConstructor{!}}\AgdaSpace{}%
\AgdaGeneralizable{ε}\AgdaSymbol{)}\<%
\\
%
\>[2]\AgdaOperator{\AgdaInductiveConstructor{\AgdaUnderscore{}·\AgdaUnderscore{}}}\AgdaSpace{}%
\AgdaSymbol{:}\AgdaSpace{}%
\AgdaDatatype{Tm}\AgdaSpace{}%
\AgdaGeneralizable{Γ}\AgdaSpace{}%
\AgdaGeneralizable{ε}\AgdaSpace{}%
\AgdaSymbol{(}\AgdaGeneralizable{α}\AgdaSpace{}%
\AgdaOperator{\AgdaInductiveConstructor{⇒}}\AgdaSpace{}%
\AgdaGeneralizable{β}\AgdaSpace{}%
\AgdaOperator{\AgdaInductiveConstructor{!}}\AgdaSpace{}%
\AgdaGeneralizable{ε}\AgdaSymbol{)}\AgdaSpace{}%
\AgdaSymbol{→}\AgdaSpace{}%
\AgdaDatatype{Tm}\AgdaSpace{}%
\AgdaGeneralizable{Γ}\AgdaSpace{}%
\AgdaGeneralizable{ε}\AgdaSpace{}%
\AgdaGeneralizable{α}\AgdaSpace{}%
\AgdaSymbol{→}\AgdaSpace{}%
\AgdaDatatype{Tm}\AgdaSpace{}%
\AgdaGeneralizable{Γ}\AgdaSpace{}%
\AgdaGeneralizable{ε}\AgdaSpace{}%
\AgdaGeneralizable{β}\<%
\\
%
\>[2]\AgdaInductiveConstructor{zero}\AgdaSpace{}%
\AgdaSymbol{:}\AgdaSpace{}%
\AgdaDatatype{Tm}\AgdaSpace{}%
\AgdaGeneralizable{Γ}\AgdaSpace{}%
\AgdaGeneralizable{ε}\AgdaSpace{}%
\AgdaInductiveConstructor{`ℕ}\<%
\\
%
\>[2]\AgdaInductiveConstructor{suc}\AgdaSpace{}%
\AgdaSymbol{:}\AgdaSpace{}%
\AgdaDatatype{Tm}\AgdaSpace{}%
\AgdaGeneralizable{Γ}\AgdaSpace{}%
\AgdaGeneralizable{ε}\AgdaSpace{}%
\AgdaInductiveConstructor{`ℕ}\AgdaSpace{}%
\AgdaSymbol{→}\AgdaSpace{}%
\AgdaDatatype{Tm}\AgdaSpace{}%
\AgdaGeneralizable{Γ}\AgdaSpace{}%
\AgdaGeneralizable{ε}\AgdaSpace{}%
\AgdaInductiveConstructor{`ℕ}\<%
\\
%
\>[2]\AgdaInductiveConstructor{case}\AgdaSpace{}%
\AgdaSymbol{:}\AgdaSpace{}%
\AgdaDatatype{Tm}\AgdaSpace{}%
\AgdaGeneralizable{Γ}\AgdaSpace{}%
\AgdaGeneralizable{ε}\AgdaSpace{}%
\AgdaInductiveConstructor{`ℕ}\AgdaSpace{}%
\AgdaSymbol{→}\AgdaSpace{}%
\AgdaDatatype{Tm}\AgdaSpace{}%
\AgdaGeneralizable{Γ}\AgdaSpace{}%
\AgdaGeneralizable{ε}\AgdaSpace{}%
\AgdaGeneralizable{α}\AgdaSpace{}%
\AgdaSymbol{→}\AgdaSpace{}%
\AgdaDatatype{Tm}\AgdaSpace{}%
\AgdaSymbol{(}\AgdaGeneralizable{Γ}\AgdaSpace{}%
\AgdaOperator{\AgdaInductiveConstructor{,}}\AgdaSpace{}%
\AgdaInductiveConstructor{`ℕ}\AgdaSymbol{)}\AgdaSpace{}%
\AgdaGeneralizable{ε}\AgdaSpace{}%
\AgdaGeneralizable{α}\AgdaSpace{}%
\AgdaSymbol{→}\AgdaSpace{}%
\AgdaDatatype{Tm}\AgdaSpace{}%
\AgdaGeneralizable{Γ}\AgdaSpace{}%
\AgdaGeneralizable{ε}\AgdaSpace{}%
\AgdaGeneralizable{α}\<%
\\
%
\>[2]\AgdaOperator{\AgdaInductiveConstructor{μ\AgdaUnderscore{}}}\AgdaSpace{}%
\AgdaSymbol{:}\AgdaSpace{}%
\AgdaDatatype{Tm}\AgdaSpace{}%
\AgdaSymbol{(}\AgdaGeneralizable{Γ}\AgdaSpace{}%
\AgdaOperator{\AgdaInductiveConstructor{,}}\AgdaSpace{}%
\AgdaGeneralizable{α}\AgdaSpace{}%
\AgdaOperator{\AgdaInductiveConstructor{⇒}}\AgdaSpace{}%
\AgdaGeneralizable{β}\AgdaSpace{}%
\AgdaOperator{\AgdaInductiveConstructor{!}}\AgdaSpace{}%
\AgdaGeneralizable{ε}\AgdaSymbol{)}\AgdaSpace{}%
\AgdaGeneralizable{ε}\AgdaSpace{}%
\AgdaSymbol{(}\AgdaGeneralizable{α}\AgdaSpace{}%
\AgdaOperator{\AgdaInductiveConstructor{⇒}}\AgdaSpace{}%
\AgdaGeneralizable{β}\AgdaSpace{}%
\AgdaOperator{\AgdaInductiveConstructor{!}}\AgdaSpace{}%
\AgdaGeneralizable{ε}\AgdaSymbol{)}\AgdaSpace{}%
\AgdaSymbol{→}\AgdaSpace{}%
\AgdaDatatype{Tm}\AgdaSpace{}%
\AgdaGeneralizable{Γ}\AgdaSpace{}%
\AgdaGeneralizable{ε}\AgdaSpace{}%
\AgdaSymbol{(}\AgdaGeneralizable{α}\AgdaSpace{}%
\AgdaOperator{\AgdaInductiveConstructor{⇒}}\AgdaSpace{}%
\AgdaGeneralizable{β}\AgdaSpace{}%
\AgdaOperator{\AgdaInductiveConstructor{!}}\AgdaSpace{}%
\AgdaGeneralizable{ε}\AgdaSymbol{)}\<%
\\
%
\>[2]\AgdaInductiveConstructor{delimit}\AgdaSpace{}%
\AgdaSymbol{:}\AgdaSpace{}%
\AgdaDatatype{Tm}\AgdaSpace{}%
\AgdaGeneralizable{Γ}\AgdaSpace{}%
\AgdaSymbol{(}\AgdaGeneralizable{ε}\AgdaSpace{}%
\AgdaOperator{\AgdaInductiveConstructor{,}}\AgdaSpace{}%
\AgdaGeneralizable{α}\AgdaSymbol{)}\AgdaSpace{}%
\AgdaGeneralizable{α}\AgdaSpace{}%
\AgdaSymbol{→}\AgdaSpace{}%
\AgdaDatatype{Tm}\AgdaSpace{}%
\AgdaGeneralizable{Γ}\AgdaSpace{}%
\AgdaGeneralizable{ε}\AgdaSpace{}%
\AgdaGeneralizable{α}\<%
\\
%
\>[2]\AgdaInductiveConstructor{grab}\AgdaSpace{}%
\AgdaSymbol{:}\AgdaSpace{}%
\AgdaDatatype{Tm}\AgdaSpace{}%
\AgdaSymbol{(}\AgdaGeneralizable{Γ}\AgdaSpace{}%
\AgdaOperator{\AgdaInductiveConstructor{,}}\AgdaSpace{}%
\AgdaGeneralizable{α}\AgdaSpace{}%
\AgdaOperator{\AgdaInductiveConstructor{⇒}}\AgdaSpace{}%
\AgdaGeneralizable{β}\AgdaSpace{}%
\AgdaOperator{\AgdaInductiveConstructor{!}}\AgdaSpace{}%
\AgdaGeneralizable{ε}\AgdaSymbol{)}\AgdaSpace{}%
\AgdaGeneralizable{ε}\AgdaSpace{}%
\AgdaGeneralizable{β}\AgdaSpace{}%
\AgdaSymbol{→}\AgdaSpace{}%
\AgdaDatatype{Tm}\AgdaSpace{}%
\AgdaGeneralizable{Γ}\AgdaSpace{}%
\AgdaSymbol{(}\AgdaGeneralizable{ε}\AgdaSpace{}%
\AgdaOperator{\AgdaInductiveConstructor{,}}\AgdaSpace{}%
\AgdaGeneralizable{β}\AgdaSymbol{)}\AgdaSpace{}%
\AgdaGeneralizable{α}\<%
\end{code}
\begin{code}[hide]%
\>[0]\AgdaFunction{length}\AgdaSpace{}%
\AgdaSymbol{:}\AgdaSpace{}%
\AgdaDatatype{Ctx}\AgdaSpace{}%
\AgdaSymbol{→}\AgdaSpace{}%
\AgdaDatatype{ℕ}\<%
\\
\>[0]\AgdaFunction{length}\AgdaSpace{}%
\AgdaInductiveConstructor{∙}\AgdaSpace{}%
\AgdaSymbol{=}\AgdaSpace{}%
\AgdaNumber{0}\<%
\\
\>[0]\AgdaFunction{length}\AgdaSpace{}%
\AgdaSymbol{(}\AgdaBound{Γ}\AgdaSpace{}%
\AgdaOperator{\AgdaInductiveConstructor{,}}\AgdaSpace{}%
\AgdaSymbol{\AgdaUnderscore{})}\AgdaSpace{}%
\AgdaSymbol{=}\AgdaSpace{}%
\AgdaInductiveConstructor{suc}\AgdaSpace{}%
\AgdaSymbol{(}\AgdaFunction{length}\AgdaSpace{}%
\AgdaBound{Γ}\AgdaSymbol{)}\<%
\\
%
\\[\AgdaEmptyExtraSkip]%
\>[0]\AgdaFunction{lookup}\AgdaSpace{}%
\AgdaSymbol{:}\AgdaSpace{}%
\AgdaSymbol{∀}\AgdaSpace{}%
\AgdaSymbol{\{}\AgdaBound{n}\AgdaSymbol{\}}\AgdaSpace{}%
\AgdaSymbol{→}\AgdaSpace{}%
\AgdaBound{n}\AgdaSpace{}%
\AgdaOperator{\AgdaFunction{<}}\AgdaSpace{}%
\AgdaFunction{length}\AgdaSpace{}%
\AgdaGeneralizable{Γ}\AgdaSpace{}%
\AgdaSymbol{→}\AgdaSpace{}%
\AgdaDatatype{Ty}\<%
\\
\>[0]\AgdaFunction{lookup}\AgdaSpace{}%
\AgdaSymbol{\{\AgdaUnderscore{}}\AgdaSpace{}%
\AgdaOperator{\AgdaInductiveConstructor{,}}\AgdaSpace{}%
\AgdaBound{α}\AgdaSymbol{\}}\AgdaSpace{}%
\AgdaSymbol{\{}\AgdaInductiveConstructor{zero}\AgdaSymbol{\}}\AgdaSpace{}%
\AgdaSymbol{(}\AgdaInductiveConstructor{s≤s}\AgdaSpace{}%
\AgdaInductiveConstructor{z≤n}\AgdaSymbol{)}\AgdaSpace{}%
\AgdaSymbol{=}\AgdaSpace{}%
\AgdaBound{α}\<%
\\
\>[0]\AgdaFunction{lookup}\AgdaSpace{}%
\AgdaSymbol{\{}\AgdaBound{Γ}\AgdaSpace{}%
\AgdaOperator{\AgdaInductiveConstructor{,}}\AgdaSpace{}%
\AgdaSymbol{\AgdaUnderscore{}\}}\AgdaSpace{}%
\AgdaSymbol{\{}\AgdaInductiveConstructor{suc}\AgdaSpace{}%
\AgdaBound{n}\AgdaSymbol{\}}\AgdaSpace{}%
\AgdaSymbol{(}\AgdaInductiveConstructor{s≤s}\AgdaSpace{}%
\AgdaBound{p}\AgdaSymbol{)}\AgdaSpace{}%
\AgdaSymbol{=}\AgdaSpace{}%
\AgdaFunction{lookup}\AgdaSpace{}%
\AgdaBound{p}\<%
\\
%
\\[\AgdaEmptyExtraSkip]%
\>[0]\AgdaFunction{count}\AgdaSpace{}%
\AgdaSymbol{:}\AgdaSpace{}%
\AgdaSymbol{∀}\AgdaSpace{}%
\AgdaSymbol{\{}\AgdaBound{n}\AgdaSymbol{\}}\AgdaSpace{}%
\AgdaSymbol{(}\AgdaBound{p}\AgdaSpace{}%
\AgdaSymbol{:}\AgdaSpace{}%
\AgdaBound{n}\AgdaSpace{}%
\AgdaOperator{\AgdaFunction{<}}\AgdaSpace{}%
\AgdaFunction{length}\AgdaSpace{}%
\AgdaGeneralizable{Γ}\AgdaSymbol{)}\AgdaSpace{}%
\AgdaSymbol{→}\AgdaSpace{}%
\AgdaGeneralizable{Γ}\AgdaSpace{}%
\AgdaOperator{\AgdaDatatype{∋}}\AgdaSpace{}%
\AgdaFunction{lookup}\AgdaSpace{}%
\AgdaBound{p}\<%
\\
\>[0]\AgdaFunction{count}\AgdaSpace{}%
\AgdaSymbol{\{\AgdaUnderscore{}}\AgdaSpace{}%
\AgdaOperator{\AgdaInductiveConstructor{,}}\AgdaSpace{}%
\AgdaSymbol{\AgdaUnderscore{}\}}\AgdaSpace{}%
\AgdaSymbol{\{}\AgdaInductiveConstructor{zero}\AgdaSymbol{\}}\AgdaSpace{}%
\AgdaSymbol{(}\AgdaInductiveConstructor{s≤s}\AgdaSpace{}%
\AgdaInductiveConstructor{z≤n}\AgdaSymbol{)}\AgdaSpace{}%
\AgdaSymbol{=}\AgdaSpace{}%
\AgdaInductiveConstructor{zero}\<%
\\
\>[0]\AgdaFunction{count}\AgdaSpace{}%
\AgdaSymbol{\{}\AgdaBound{Γ}\AgdaSpace{}%
\AgdaOperator{\AgdaInductiveConstructor{,}}\AgdaSpace{}%
\AgdaSymbol{\AgdaUnderscore{}\}}\AgdaSpace{}%
\AgdaSymbol{\{}\AgdaInductiveConstructor{suc}\AgdaSpace{}%
\AgdaBound{n}\AgdaSymbol{\}}\AgdaSpace{}%
\AgdaSymbol{(}\AgdaInductiveConstructor{s≤s}\AgdaSpace{}%
\AgdaBound{p}\AgdaSymbol{)}\AgdaSpace{}%
\AgdaSymbol{=}\AgdaSpace{}%
\AgdaInductiveConstructor{suc}\AgdaSpace{}%
\AgdaSymbol{(}\AgdaFunction{count}\AgdaSpace{}%
\AgdaBound{p}\AgdaSymbol{)}\<%
\\
%
\\[\AgdaEmptyExtraSkip]%
\>[0]\AgdaFunction{var\#}\AgdaSpace{}%
\AgdaSymbol{:}\AgdaSpace{}%
\AgdaSymbol{∀}\AgdaSpace{}%
\AgdaBound{n}\AgdaSpace{}%
\AgdaSymbol{\{}\AgdaBound{n∈Γ}\AgdaSpace{}%
\AgdaSymbol{:}\AgdaSpace{}%
\AgdaFunction{True}\AgdaSpace{}%
\AgdaSymbol{(}\AgdaInductiveConstructor{suc}\AgdaSpace{}%
\AgdaBound{n}\AgdaSpace{}%
\AgdaOperator{\AgdaFunction{≤?}}\AgdaSpace{}%
\AgdaFunction{length}\AgdaSpace{}%
\AgdaGeneralizable{Γ}\AgdaSymbol{)\}}\AgdaSpace{}%
\AgdaSymbol{→}\AgdaSpace{}%
\AgdaDatatype{Tm}\AgdaSpace{}%
\AgdaGeneralizable{Γ}\AgdaSpace{}%
\AgdaGeneralizable{ε}\AgdaSpace{}%
\AgdaSymbol{(}\AgdaFunction{lookup}\AgdaSpace{}%
\AgdaSymbol{(}\AgdaFunction{toWitness}\AgdaSpace{}%
\AgdaBound{n∈Γ}\AgdaSymbol{))}\<%
\\
\>[0]\AgdaFunction{var\#}\AgdaSpace{}%
\AgdaBound{n}\AgdaSpace{}%
\AgdaSymbol{\{}\AgdaBound{n∈Γ}\AgdaSymbol{\}}\AgdaSpace{}%
\AgdaSymbol{=}\AgdaSpace{}%
\AgdaInductiveConstructor{var}\AgdaSpace{}%
\AgdaSymbol{(}\AgdaFunction{count}\AgdaSpace{}%
\AgdaSymbol{(}\AgdaFunction{toWitness}\AgdaSpace{}%
\AgdaBound{n∈Γ}\AgdaSymbol{))}\<%
\end{code}

\subsection{Renaming and Subsitution}

\begin{code}%
\>[0]\AgdaFunction{Ren}\AgdaSpace{}%
\AgdaSymbol{:}\AgdaSpace{}%
\AgdaDatatype{Ctx}\AgdaSpace{}%
\AgdaSymbol{→}\AgdaSpace{}%
\AgdaDatatype{Ctx}\AgdaSpace{}%
\AgdaSymbol{→}\AgdaSpace{}%
\AgdaPrimitive{Set}\<%
\\
\>[0]\AgdaFunction{Ren}\AgdaSpace{}%
\AgdaBound{Γ}\AgdaSpace{}%
\AgdaBound{Δ}\AgdaSpace{}%
\AgdaSymbol{=}\AgdaSpace{}%
\AgdaSymbol{∀}\AgdaSpace{}%
\AgdaSymbol{\{}\AgdaBound{α}\AgdaSymbol{\}}\AgdaSpace{}%
\AgdaSymbol{→}\AgdaSpace{}%
\AgdaBound{Γ}\AgdaSpace{}%
\AgdaOperator{\AgdaDatatype{∋}}\AgdaSpace{}%
\AgdaBound{α}\AgdaSpace{}%
\AgdaSymbol{→}\AgdaSpace{}%
\AgdaBound{Δ}\AgdaSpace{}%
\AgdaOperator{\AgdaDatatype{∋}}\AgdaSpace{}%
\AgdaBound{α}\<%
\\
%
\\[\AgdaEmptyExtraSkip]%
\>[0]\AgdaFunction{ext}\AgdaSpace{}%
\AgdaSymbol{:}\AgdaSpace{}%
\AgdaFunction{Ren}\AgdaSpace{}%
\AgdaGeneralizable{Γ}\AgdaSpace{}%
\AgdaGeneralizable{Δ}\AgdaSpace{}%
\AgdaSymbol{→}\AgdaSpace{}%
\AgdaFunction{Ren}\AgdaSpace{}%
\AgdaSymbol{(}\AgdaGeneralizable{Γ}\AgdaSpace{}%
\AgdaOperator{\AgdaInductiveConstructor{,}}\AgdaSpace{}%
\AgdaGeneralizable{α}\AgdaSymbol{)}\AgdaSpace{}%
\AgdaSymbol{(}\AgdaGeneralizable{Δ}\AgdaSpace{}%
\AgdaOperator{\AgdaInductiveConstructor{,}}\AgdaSpace{}%
\AgdaGeneralizable{α}\AgdaSymbol{)}\<%
\\
\>[0]\AgdaFunction{ext}\AgdaSpace{}%
\AgdaBound{ρ}\AgdaSpace{}%
\AgdaInductiveConstructor{zero}\AgdaSpace{}%
\AgdaSymbol{=}\AgdaSpace{}%
\AgdaInductiveConstructor{zero}\<%
\\
\>[0]\AgdaFunction{ext}\AgdaSpace{}%
\AgdaBound{ρ}\AgdaSpace{}%
\AgdaSymbol{(}\AgdaInductiveConstructor{suc}\AgdaSpace{}%
\AgdaBound{i}\AgdaSymbol{)}\AgdaSpace{}%
\AgdaSymbol{=}\AgdaSpace{}%
\AgdaInductiveConstructor{suc}\AgdaSpace{}%
\AgdaSymbol{(}\AgdaBound{ρ}\AgdaSpace{}%
\AgdaBound{i}\AgdaSymbol{)}\<%
\\
%
\\[\AgdaEmptyExtraSkip]%
\>[0]\AgdaFunction{ren}\AgdaSpace{}%
\AgdaSymbol{:}\AgdaSpace{}%
\AgdaFunction{Ren}\AgdaSpace{}%
\AgdaGeneralizable{Γ}\AgdaSpace{}%
\AgdaGeneralizable{Δ}\AgdaSpace{}%
\AgdaSymbol{→}\AgdaSpace{}%
\AgdaDatatype{Tm}\AgdaSpace{}%
\AgdaGeneralizable{Γ}\AgdaSpace{}%
\AgdaGeneralizable{ε}\AgdaSpace{}%
\AgdaGeneralizable{α}\AgdaSpace{}%
\AgdaSymbol{→}\AgdaSpace{}%
\AgdaDatatype{Tm}\AgdaSpace{}%
\AgdaGeneralizable{Δ}\AgdaSpace{}%
\AgdaGeneralizable{ε}\AgdaSpace{}%
\AgdaGeneralizable{α}\<%
\\
\>[0]\AgdaFunction{ren}\AgdaSpace{}%
\AgdaBound{ρ}\AgdaSpace{}%
\AgdaSymbol{(}\AgdaInductiveConstructor{var}\AgdaSpace{}%
\AgdaBound{i}\AgdaSymbol{)}\AgdaSpace{}%
\AgdaSymbol{=}\AgdaSpace{}%
\AgdaInductiveConstructor{var}\AgdaSpace{}%
\AgdaSymbol{(}\AgdaBound{ρ}\AgdaSpace{}%
\AgdaBound{i}\AgdaSymbol{)}\<%
\\
\>[0]\AgdaFunction{ren}\AgdaSpace{}%
\AgdaBound{ρ}\AgdaSpace{}%
\AgdaSymbol{(}\AgdaOperator{\AgdaInductiveConstructor{ƛ}}\AgdaSpace{}%
\AgdaBound{t}\AgdaSymbol{)}\AgdaSpace{}%
\AgdaSymbol{=}\AgdaSpace{}%
\AgdaOperator{\AgdaInductiveConstructor{ƛ}}\AgdaSpace{}%
\AgdaFunction{ren}\AgdaSpace{}%
\AgdaSymbol{(}\AgdaFunction{ext}\AgdaSpace{}%
\AgdaBound{ρ}\AgdaSymbol{)}\AgdaSpace{}%
\AgdaBound{t}\<%
\\
\>[0]\AgdaFunction{ren}\AgdaSpace{}%
\AgdaBound{ρ}\AgdaSpace{}%
\AgdaSymbol{(}\AgdaBound{t}\AgdaSpace{}%
\AgdaOperator{\AgdaInductiveConstructor{·}}\AgdaSpace{}%
\AgdaBound{u}\AgdaSymbol{)}\AgdaSpace{}%
\AgdaSymbol{=}\AgdaSpace{}%
\AgdaFunction{ren}\AgdaSpace{}%
\AgdaBound{ρ}\AgdaSpace{}%
\AgdaBound{t}\AgdaSpace{}%
\AgdaOperator{\AgdaInductiveConstructor{·}}\AgdaSpace{}%
\AgdaFunction{ren}\AgdaSpace{}%
\AgdaBound{ρ}\AgdaSpace{}%
\AgdaBound{u}\<%
\\
\>[0]\AgdaFunction{ren}\AgdaSpace{}%
\AgdaBound{ρ}\AgdaSpace{}%
\AgdaInductiveConstructor{zero}\AgdaSpace{}%
\AgdaSymbol{=}\AgdaSpace{}%
\AgdaInductiveConstructor{zero}\<%
\\
\>[0]\AgdaFunction{ren}\AgdaSpace{}%
\AgdaBound{ρ}\AgdaSpace{}%
\AgdaSymbol{(}\AgdaInductiveConstructor{suc}\AgdaSpace{}%
\AgdaBound{t}\AgdaSymbol{)}\AgdaSpace{}%
\AgdaSymbol{=}\AgdaSpace{}%
\AgdaInductiveConstructor{suc}\AgdaSpace{}%
\AgdaSymbol{(}\AgdaFunction{ren}\AgdaSpace{}%
\AgdaBound{ρ}\AgdaSpace{}%
\AgdaBound{t}\AgdaSymbol{)}\<%
\\
\>[0]\AgdaFunction{ren}\AgdaSpace{}%
\AgdaBound{ρ}\AgdaSpace{}%
\AgdaSymbol{(}\AgdaInductiveConstructor{case}\AgdaSpace{}%
\AgdaBound{n}\AgdaSpace{}%
\AgdaBound{z}\AgdaSpace{}%
\AgdaBound{s}\AgdaSymbol{)}\AgdaSpace{}%
\AgdaSymbol{=}\AgdaSpace{}%
\AgdaInductiveConstructor{case}\AgdaSpace{}%
\AgdaSymbol{(}\AgdaFunction{ren}\AgdaSpace{}%
\AgdaBound{ρ}\AgdaSpace{}%
\AgdaBound{n}\AgdaSymbol{)}\AgdaSpace{}%
\AgdaSymbol{(}\AgdaFunction{ren}\AgdaSpace{}%
\AgdaBound{ρ}\AgdaSpace{}%
\AgdaBound{z}\AgdaSymbol{)}\AgdaSpace{}%
\AgdaSymbol{(}\AgdaFunction{ren}\AgdaSpace{}%
\AgdaSymbol{(}\AgdaFunction{ext}\AgdaSpace{}%
\AgdaBound{ρ}\AgdaSymbol{)}\AgdaSpace{}%
\AgdaBound{s}\AgdaSymbol{)}\<%
\\
\>[0]\AgdaFunction{ren}\AgdaSpace{}%
\AgdaBound{ρ}\AgdaSpace{}%
\AgdaSymbol{(}\AgdaOperator{\AgdaInductiveConstructor{μ}}\AgdaSpace{}%
\AgdaBound{t}\AgdaSymbol{)}\AgdaSpace{}%
\AgdaSymbol{=}\AgdaSpace{}%
\AgdaOperator{\AgdaInductiveConstructor{μ}}\AgdaSpace{}%
\AgdaFunction{ren}\AgdaSpace{}%
\AgdaSymbol{(}\AgdaFunction{ext}\AgdaSpace{}%
\AgdaBound{ρ}\AgdaSymbol{)}\AgdaSpace{}%
\AgdaBound{t}\<%
\\
\>[0]\AgdaFunction{ren}\AgdaSpace{}%
\AgdaBound{ρ}\AgdaSpace{}%
\AgdaSymbol{(}\AgdaInductiveConstructor{delimit}\AgdaSpace{}%
\AgdaBound{t}\AgdaSymbol{)}\AgdaSpace{}%
\AgdaSymbol{=}\AgdaSpace{}%
\AgdaInductiveConstructor{delimit}\AgdaSpace{}%
\AgdaSymbol{(}\AgdaFunction{ren}\AgdaSpace{}%
\AgdaBound{ρ}\AgdaSpace{}%
\AgdaBound{t}\AgdaSymbol{)}\<%
\\
\>[0]\AgdaFunction{ren}\AgdaSpace{}%
\AgdaBound{ρ}\AgdaSpace{}%
\AgdaSymbol{(}\AgdaInductiveConstructor{grab}\AgdaSpace{}%
\AgdaBound{t}\AgdaSymbol{)}\AgdaSpace{}%
\AgdaSymbol{=}\AgdaSpace{}%
\AgdaInductiveConstructor{grab}\AgdaSpace{}%
\AgdaSymbol{(}\AgdaFunction{ren}\AgdaSpace{}%
\AgdaSymbol{(}\AgdaFunction{ext}\AgdaSpace{}%
\AgdaBound{ρ}\AgdaSymbol{)}\AgdaSpace{}%
\AgdaBound{t}\AgdaSymbol{)}\<%
\\
%
\\[\AgdaEmptyExtraSkip]%
\>[0]\AgdaOperator{\AgdaFunction{↑\AgdaUnderscore{}}}\AgdaSpace{}%
\AgdaSymbol{:}\AgdaSpace{}%
\AgdaDatatype{Tm}\AgdaSpace{}%
\AgdaGeneralizable{Γ}\AgdaSpace{}%
\AgdaGeneralizable{ε}\AgdaSpace{}%
\AgdaGeneralizable{α}\AgdaSpace{}%
\AgdaSymbol{→}\AgdaSpace{}%
\AgdaDatatype{Tm}\AgdaSpace{}%
\AgdaSymbol{(}\AgdaGeneralizable{Γ}\AgdaSpace{}%
\AgdaOperator{\AgdaInductiveConstructor{,}}\AgdaSpace{}%
\AgdaGeneralizable{β}\AgdaSymbol{)}\AgdaSpace{}%
\AgdaGeneralizable{ε}\AgdaSpace{}%
\AgdaGeneralizable{α}\<%
\\
\>[0]\AgdaOperator{\AgdaFunction{↑\AgdaUnderscore{}}}\AgdaSpace{}%
\AgdaSymbol{=}\AgdaSpace{}%
\AgdaFunction{ren}\AgdaSpace{}%
\AgdaInductiveConstructor{suc}\<%
\end{code}

\begin{code}%
\>[0]\AgdaFunction{Sub}\AgdaSpace{}%
\AgdaSymbol{:}\AgdaSpace{}%
\AgdaDatatype{Ctx}\AgdaSpace{}%
\AgdaSymbol{→}\AgdaSpace{}%
\AgdaDatatype{Ctx}\AgdaSpace{}%
\AgdaSymbol{→}\AgdaSpace{}%
\AgdaPrimitive{Set}\<%
\\
\>[0]\AgdaFunction{Sub}\AgdaSpace{}%
\AgdaBound{Γ}\AgdaSpace{}%
\AgdaBound{Δ}\AgdaSpace{}%
\AgdaSymbol{=}\AgdaSpace{}%
\AgdaSymbol{∀}\AgdaSpace{}%
\AgdaSymbol{\{}\AgdaBound{α}\AgdaSymbol{\}}\AgdaSpace{}%
\AgdaSymbol{→}\AgdaSpace{}%
\AgdaBound{Γ}\AgdaSpace{}%
\AgdaOperator{\AgdaDatatype{∋}}\AgdaSpace{}%
\AgdaBound{α}\AgdaSpace{}%
\AgdaSymbol{→}\AgdaSpace{}%
\AgdaSymbol{(∀}\AgdaSpace{}%
\AgdaSymbol{\{}\AgdaBound{ε}\AgdaSymbol{\}}\AgdaSpace{}%
\AgdaSymbol{→}\AgdaSpace{}%
\AgdaDatatype{Tm}\AgdaSpace{}%
\AgdaBound{Δ}\AgdaSpace{}%
\AgdaBound{ε}\AgdaSpace{}%
\AgdaBound{α}\AgdaSymbol{)}\<%
\\
%
\\[\AgdaEmptyExtraSkip]%
\>[0]\AgdaFunction{exts}\AgdaSpace{}%
\AgdaSymbol{:}\AgdaSpace{}%
\AgdaFunction{Sub}\AgdaSpace{}%
\AgdaGeneralizable{Γ}\AgdaSpace{}%
\AgdaGeneralizable{Δ}\AgdaSpace{}%
\AgdaSymbol{→}\AgdaSpace{}%
\AgdaFunction{Sub}\AgdaSpace{}%
\AgdaSymbol{(}\AgdaGeneralizable{Γ}\AgdaSpace{}%
\AgdaOperator{\AgdaInductiveConstructor{,}}\AgdaSpace{}%
\AgdaGeneralizable{α}\AgdaSymbol{)}\AgdaSpace{}%
\AgdaSymbol{(}\AgdaGeneralizable{Δ}\AgdaSpace{}%
\AgdaOperator{\AgdaInductiveConstructor{,}}\AgdaSpace{}%
\AgdaGeneralizable{α}\AgdaSymbol{)}\<%
\\
\>[0]\AgdaFunction{exts}\AgdaSpace{}%
\AgdaBound{σ}\AgdaSpace{}%
\AgdaInductiveConstructor{zero}\AgdaSpace{}%
\AgdaSymbol{=}\AgdaSpace{}%
\AgdaInductiveConstructor{var}\AgdaSpace{}%
\AgdaInductiveConstructor{zero}\<%
\\
\>[0]\AgdaFunction{exts}\AgdaSpace{}%
\AgdaBound{σ}\AgdaSpace{}%
\AgdaSymbol{(}\AgdaInductiveConstructor{suc}\AgdaSpace{}%
\AgdaBound{i}\AgdaSymbol{)}\AgdaSpace{}%
\AgdaSymbol{=}\AgdaSpace{}%
\AgdaOperator{\AgdaFunction{↑}}\AgdaSpace{}%
\AgdaSymbol{(}\AgdaBound{σ}\AgdaSpace{}%
\AgdaBound{i}\AgdaSymbol{)}\<%
\\
%
\\[\AgdaEmptyExtraSkip]%
\>[0]\AgdaFunction{sub}\AgdaSpace{}%
\AgdaSymbol{:}\AgdaSpace{}%
\AgdaFunction{Sub}\AgdaSpace{}%
\AgdaGeneralizable{Γ}\AgdaSpace{}%
\AgdaGeneralizable{Δ}\AgdaSpace{}%
\AgdaSymbol{→}\AgdaSpace{}%
\AgdaDatatype{Tm}\AgdaSpace{}%
\AgdaGeneralizable{Γ}\AgdaSpace{}%
\AgdaGeneralizable{ε}\AgdaSpace{}%
\AgdaGeneralizable{α}\AgdaSpace{}%
\AgdaSymbol{→}\AgdaSpace{}%
\AgdaDatatype{Tm}\AgdaSpace{}%
\AgdaGeneralizable{Δ}\AgdaSpace{}%
\AgdaGeneralizable{ε}\AgdaSpace{}%
\AgdaGeneralizable{α}\<%
\\
\>[0]\AgdaFunction{sub}\AgdaSpace{}%
\AgdaBound{σ}\AgdaSpace{}%
\AgdaSymbol{(}\AgdaInductiveConstructor{var}\AgdaSpace{}%
\AgdaBound{i}\AgdaSymbol{)}\AgdaSpace{}%
\AgdaSymbol{=}\AgdaSpace{}%
\AgdaBound{σ}\AgdaSpace{}%
\AgdaBound{i}\<%
\\
\>[0]\AgdaFunction{sub}\AgdaSpace{}%
\AgdaBound{σ}\AgdaSpace{}%
\AgdaSymbol{(}\AgdaOperator{\AgdaInductiveConstructor{ƛ}}\AgdaSpace{}%
\AgdaBound{t}\AgdaSymbol{)}\AgdaSpace{}%
\AgdaSymbol{=}\AgdaSpace{}%
\AgdaOperator{\AgdaInductiveConstructor{ƛ}}\AgdaSpace{}%
\AgdaFunction{sub}\AgdaSpace{}%
\AgdaSymbol{(}\AgdaFunction{exts}\AgdaSpace{}%
\AgdaBound{σ}\AgdaSymbol{)}\AgdaSpace{}%
\AgdaBound{t}\<%
\\
\>[0]\AgdaFunction{sub}\AgdaSpace{}%
\AgdaBound{σ}\AgdaSpace{}%
\AgdaSymbol{(}\AgdaBound{t}\AgdaSpace{}%
\AgdaOperator{\AgdaInductiveConstructor{·}}\AgdaSpace{}%
\AgdaBound{u}\AgdaSymbol{)}\AgdaSpace{}%
\AgdaSymbol{=}\AgdaSpace{}%
\AgdaFunction{sub}\AgdaSpace{}%
\AgdaBound{σ}\AgdaSpace{}%
\AgdaBound{t}\AgdaSpace{}%
\AgdaOperator{\AgdaInductiveConstructor{·}}\AgdaSpace{}%
\AgdaFunction{sub}\AgdaSpace{}%
\AgdaBound{σ}\AgdaSpace{}%
\AgdaBound{u}\<%
\\
\>[0]\AgdaFunction{sub}\AgdaSpace{}%
\AgdaBound{σ}\AgdaSpace{}%
\AgdaInductiveConstructor{zero}\AgdaSpace{}%
\AgdaSymbol{=}\AgdaSpace{}%
\AgdaInductiveConstructor{zero}\<%
\\
\>[0]\AgdaFunction{sub}\AgdaSpace{}%
\AgdaBound{σ}\AgdaSpace{}%
\AgdaSymbol{(}\AgdaInductiveConstructor{suc}\AgdaSpace{}%
\AgdaBound{t}\AgdaSymbol{)}\AgdaSpace{}%
\AgdaSymbol{=}\AgdaSpace{}%
\AgdaInductiveConstructor{suc}\AgdaSpace{}%
\AgdaSymbol{(}\AgdaFunction{sub}\AgdaSpace{}%
\AgdaBound{σ}\AgdaSpace{}%
\AgdaBound{t}\AgdaSymbol{)}\<%
\\
\>[0]\AgdaFunction{sub}\AgdaSpace{}%
\AgdaBound{σ}\AgdaSpace{}%
\AgdaSymbol{(}\AgdaInductiveConstructor{case}\AgdaSpace{}%
\AgdaBound{n}\AgdaSpace{}%
\AgdaBound{z}\AgdaSpace{}%
\AgdaBound{s}\AgdaSymbol{)}\AgdaSpace{}%
\AgdaSymbol{=}\AgdaSpace{}%
\AgdaInductiveConstructor{case}\AgdaSpace{}%
\AgdaSymbol{(}\AgdaFunction{sub}\AgdaSpace{}%
\AgdaBound{σ}\AgdaSpace{}%
\AgdaBound{n}\AgdaSymbol{)}\AgdaSpace{}%
\AgdaSymbol{(}\AgdaFunction{sub}\AgdaSpace{}%
\AgdaBound{σ}\AgdaSpace{}%
\AgdaBound{z}\AgdaSymbol{)}\AgdaSpace{}%
\AgdaSymbol{(}\AgdaFunction{sub}\AgdaSpace{}%
\AgdaSymbol{(}\AgdaFunction{exts}\AgdaSpace{}%
\AgdaBound{σ}\AgdaSymbol{)}\AgdaSpace{}%
\AgdaBound{s}\AgdaSymbol{)}\<%
\\
\>[0]\AgdaFunction{sub}\AgdaSpace{}%
\AgdaBound{σ}\AgdaSpace{}%
\AgdaSymbol{(}\AgdaOperator{\AgdaInductiveConstructor{μ}}\AgdaSpace{}%
\AgdaBound{t}\AgdaSymbol{)}\AgdaSpace{}%
\AgdaSymbol{=}\AgdaSpace{}%
\AgdaOperator{\AgdaInductiveConstructor{μ}}\AgdaSpace{}%
\AgdaFunction{sub}\AgdaSpace{}%
\AgdaSymbol{(}\AgdaFunction{exts}\AgdaSpace{}%
\AgdaBound{σ}\AgdaSymbol{)}\AgdaSpace{}%
\AgdaBound{t}\<%
\\
\>[0]\AgdaFunction{sub}\AgdaSpace{}%
\AgdaBound{σ}\AgdaSpace{}%
\AgdaSymbol{(}\AgdaInductiveConstructor{delimit}\AgdaSpace{}%
\AgdaBound{t}\AgdaSymbol{)}\AgdaSpace{}%
\AgdaSymbol{=}\AgdaSpace{}%
\AgdaInductiveConstructor{delimit}\AgdaSpace{}%
\AgdaSymbol{(}\AgdaFunction{sub}\AgdaSpace{}%
\AgdaBound{σ}\AgdaSpace{}%
\AgdaBound{t}\AgdaSymbol{)}\<%
\\
\>[0]\AgdaFunction{sub}\AgdaSpace{}%
\AgdaBound{σ}\AgdaSpace{}%
\AgdaSymbol{(}\AgdaInductiveConstructor{grab}\AgdaSpace{}%
\AgdaBound{t}\AgdaSymbol{)}\AgdaSpace{}%
\AgdaSymbol{=}\AgdaSpace{}%
\AgdaInductiveConstructor{grab}\AgdaSpace{}%
\AgdaSymbol{(}\AgdaFunction{sub}\AgdaSpace{}%
\AgdaSymbol{(}\AgdaFunction{exts}\AgdaSpace{}%
\AgdaBound{σ}\AgdaSymbol{)}\AgdaSpace{}%
\AgdaBound{t}\AgdaSymbol{)}\<%
\\
%
\\[\AgdaEmptyExtraSkip]%
\>[0]\AgdaOperator{\AgdaFunction{\AgdaUnderscore{}[\AgdaUnderscore{}]}}\AgdaSpace{}%
\AgdaSymbol{:}\AgdaSpace{}%
\AgdaDatatype{Tm}\AgdaSpace{}%
\AgdaSymbol{(}\AgdaGeneralizable{Γ}\AgdaSpace{}%
\AgdaOperator{\AgdaInductiveConstructor{,}}\AgdaSpace{}%
\AgdaGeneralizable{α}\AgdaSymbol{)}\AgdaSpace{}%
\AgdaGeneralizable{ε}\AgdaSpace{}%
\AgdaGeneralizable{β}\AgdaSpace{}%
\AgdaSymbol{→}\AgdaSpace{}%
\AgdaSymbol{(∀}\AgdaSpace{}%
\AgdaSymbol{\{}\AgdaBound{ε'}\AgdaSymbol{\}}\AgdaSpace{}%
\AgdaSymbol{→}\AgdaSpace{}%
\AgdaDatatype{Tm}\AgdaSpace{}%
\AgdaGeneralizable{Γ}\AgdaSpace{}%
\AgdaBound{ε'}\AgdaSpace{}%
\AgdaGeneralizable{α}\AgdaSymbol{)}\AgdaSpace{}%
\AgdaSymbol{→}\AgdaSpace{}%
\AgdaDatatype{Tm}\AgdaSpace{}%
\AgdaGeneralizable{Γ}\AgdaSpace{}%
\AgdaGeneralizable{ε}\AgdaSpace{}%
\AgdaGeneralizable{β}\<%
\\
\>[0]\AgdaBound{t}\AgdaSpace{}%
\AgdaOperator{\AgdaFunction{[}}\AgdaSpace{}%
\AgdaBound{u}\AgdaSpace{}%
\AgdaOperator{\AgdaFunction{]}}\AgdaSpace{}%
\AgdaSymbol{=}\AgdaSpace{}%
\AgdaFunction{sub}\AgdaSpace{}%
\AgdaSymbol{(λ}\AgdaSpace{}%
\AgdaSymbol{\{}\AgdaSpace{}%
\AgdaInductiveConstructor{zero}\AgdaSpace{}%
\AgdaSymbol{→}\AgdaSpace{}%
\AgdaBound{u}\AgdaSymbol{;}\AgdaSpace{}%
\AgdaSymbol{(}\AgdaInductiveConstructor{suc}\AgdaSpace{}%
\AgdaBound{i}\AgdaSymbol{)}\AgdaSpace{}%
\AgdaSymbol{→}\AgdaSpace{}%
\AgdaInductiveConstructor{var}\AgdaSpace{}%
\AgdaBound{i}\AgdaSpace{}%
\AgdaSymbol{\})}\AgdaSpace{}%
\AgdaBound{t}\<%
\end{code}

\begin{code}%
\>[0]\AgdaKeyword{data}\AgdaSpace{}%
\AgdaDatatype{Value}\AgdaSpace{}%
\AgdaSymbol{:}\AgdaSpace{}%
\AgdaDatatype{Tm}\AgdaSpace{}%
\AgdaGeneralizable{Γ}\AgdaSpace{}%
\AgdaGeneralizable{ε}\AgdaSpace{}%
\AgdaGeneralizable{α}\AgdaSpace{}%
\AgdaSymbol{→}\AgdaSpace{}%
\AgdaPrimitive{Set}\AgdaSpace{}%
\AgdaKeyword{where}\<%
\\
\>[0][@{}l@{\AgdaIndent{0}}]%
\>[2]\AgdaOperator{\AgdaInductiveConstructor{ƛ\AgdaUnderscore{}}}\AgdaSpace{}%
\AgdaSymbol{:}\AgdaSpace{}%
\AgdaSymbol{(}\AgdaBound{t}\AgdaSpace{}%
\AgdaSymbol{:}\AgdaSpace{}%
\AgdaDatatype{Tm}\AgdaSpace{}%
\AgdaSymbol{(}\AgdaGeneralizable{Γ}\AgdaSpace{}%
\AgdaOperator{\AgdaInductiveConstructor{,}}\AgdaSpace{}%
\AgdaGeneralizable{α}\AgdaSymbol{)}\AgdaSpace{}%
\AgdaGeneralizable{ε}\AgdaSpace{}%
\AgdaGeneralizable{β}\AgdaSymbol{)}\AgdaSpace{}%
\AgdaSymbol{→}\AgdaSpace{}%
\AgdaDatatype{Value}\AgdaSpace{}%
\AgdaSymbol{\{}\AgdaGeneralizable{Γ}\AgdaSymbol{\}}\AgdaSpace{}%
\AgdaSymbol{\{}\AgdaGeneralizable{ε'}\AgdaSymbol{\}}\AgdaSpace{}%
\AgdaSymbol{(}\AgdaOperator{\AgdaInductiveConstructor{ƛ}}\AgdaSpace{}%
\AgdaBound{t}\AgdaSymbol{)}\<%
\\
%
\>[2]\AgdaInductiveConstructor{zero}\AgdaSpace{}%
\AgdaSymbol{:}\AgdaSpace{}%
\AgdaDatatype{Value}\AgdaSpace{}%
\AgdaSymbol{\{}\AgdaGeneralizable{Γ}\AgdaSymbol{\}}\AgdaSpace{}%
\AgdaSymbol{\{}\AgdaGeneralizable{ε}\AgdaSymbol{\}}\AgdaSpace{}%
\AgdaInductiveConstructor{zero}\<%
\\
%
\>[2]\AgdaInductiveConstructor{suc}\AgdaSpace{}%
\AgdaSymbol{:}\AgdaSpace{}%
\AgdaSymbol{\{}\AgdaBound{t}\AgdaSpace{}%
\AgdaSymbol{:}\AgdaSpace{}%
\AgdaDatatype{Tm}\AgdaSpace{}%
\AgdaGeneralizable{Γ}\AgdaSpace{}%
\AgdaGeneralizable{ε}\AgdaSpace{}%
\AgdaInductiveConstructor{`ℕ}\AgdaSymbol{\}}\AgdaSpace{}%
\AgdaSymbol{→}\AgdaSpace{}%
\AgdaDatatype{Value}\AgdaSpace{}%
\AgdaBound{t}\AgdaSpace{}%
\AgdaSymbol{→}\AgdaSpace{}%
\AgdaDatatype{Value}\AgdaSpace{}%
\AgdaSymbol{(}\AgdaInductiveConstructor{suc}\AgdaSpace{}%
\AgdaBound{t}\AgdaSymbol{)}\<%
\\
%
\\[\AgdaEmptyExtraSkip]%
\>[0]\AgdaFunction{renV}\AgdaSpace{}%
\AgdaSymbol{:}\AgdaSpace{}%
\AgdaSymbol{(}\AgdaBound{ρ}\AgdaSpace{}%
\AgdaSymbol{:}\AgdaSpace{}%
\AgdaFunction{Ren}\AgdaSpace{}%
\AgdaGeneralizable{Γ}\AgdaSpace{}%
\AgdaGeneralizable{Δ}\AgdaSymbol{)}\AgdaSpace{}%
\AgdaSymbol{\{}\AgdaBound{t}\AgdaSpace{}%
\AgdaSymbol{:}\AgdaSpace{}%
\AgdaDatatype{Tm}\AgdaSpace{}%
\AgdaGeneralizable{Γ}\AgdaSpace{}%
\AgdaGeneralizable{ε}\AgdaSpace{}%
\AgdaGeneralizable{α}\AgdaSymbol{\}}\AgdaSpace{}%
\AgdaSymbol{→}\AgdaSpace{}%
\AgdaDatatype{Value}\AgdaSpace{}%
\AgdaBound{t}\AgdaSpace{}%
\AgdaSymbol{→}\AgdaSpace{}%
\AgdaDatatype{Value}\AgdaSpace{}%
\AgdaSymbol{(}\AgdaFunction{ren}\AgdaSpace{}%
\AgdaBound{ρ}\AgdaSpace{}%
\AgdaBound{t}\AgdaSymbol{)}\<%
\\
\>[0]\AgdaFunction{renV}\AgdaSpace{}%
\AgdaBound{ρ}\AgdaSpace{}%
\AgdaSymbol{(}\AgdaOperator{\AgdaInductiveConstructor{ƛ}}\AgdaSpace{}%
\AgdaBound{t}\AgdaSymbol{)}\AgdaSpace{}%
\AgdaSymbol{=}\AgdaSpace{}%
\AgdaOperator{\AgdaInductiveConstructor{ƛ}}\AgdaSpace{}%
\AgdaFunction{ren}\AgdaSpace{}%
\AgdaSymbol{(}\AgdaFunction{ext}\AgdaSpace{}%
\AgdaBound{ρ}\AgdaSymbol{)}\AgdaSpace{}%
\AgdaBound{t}\<%
\\
\>[0]\AgdaFunction{renV}\AgdaSpace{}%
\AgdaBound{ρ}\AgdaSpace{}%
\AgdaInductiveConstructor{zero}\AgdaSpace{}%
\AgdaSymbol{=}\AgdaSpace{}%
\AgdaInductiveConstructor{zero}\<%
\\
\>[0]\AgdaFunction{renV}\AgdaSpace{}%
\AgdaBound{ρ}\AgdaSpace{}%
\AgdaSymbol{(}\AgdaInductiveConstructor{suc}\AgdaSpace{}%
\AgdaBound{v}\AgdaSymbol{)}\AgdaSpace{}%
\AgdaSymbol{=}\AgdaSpace{}%
\AgdaInductiveConstructor{suc}\AgdaSpace{}%
\AgdaSymbol{(}\AgdaFunction{renV}\AgdaSpace{}%
\AgdaBound{ρ}\AgdaSpace{}%
\AgdaBound{v}\AgdaSymbol{)}\<%
\\
%
\\[\AgdaEmptyExtraSkip]%
\>[0]\AgdaFunction{↑V}\AgdaSpace{}%
\AgdaSymbol{:}\AgdaSpace{}%
\AgdaSymbol{\{}\AgdaBound{t}\AgdaSpace{}%
\AgdaSymbol{:}\AgdaSpace{}%
\AgdaDatatype{Tm}\AgdaSpace{}%
\AgdaGeneralizable{Γ}\AgdaSpace{}%
\AgdaGeneralizable{ε}\AgdaSpace{}%
\AgdaGeneralizable{α}\AgdaSymbol{\}}\AgdaSpace{}%
\AgdaSymbol{→}\AgdaSpace{}%
\AgdaDatatype{Value}\AgdaSpace{}%
\AgdaBound{t}\AgdaSpace{}%
\AgdaSymbol{→}\AgdaSpace{}%
\AgdaDatatype{Value}\AgdaSpace{}%
\AgdaSymbol{\{}\AgdaArgument{Γ}\AgdaSpace{}%
\AgdaSymbol{=}\AgdaSpace{}%
\AgdaGeneralizable{Γ}\AgdaSpace{}%
\AgdaOperator{\AgdaInductiveConstructor{,}}\AgdaSpace{}%
\AgdaGeneralizable{β}\AgdaSymbol{\}}\AgdaSpace{}%
\AgdaSymbol{(}\AgdaOperator{\AgdaFunction{↑}}\AgdaSpace{}%
\AgdaBound{t}\AgdaSymbol{)}\<%
\\
\>[0]\AgdaFunction{↑V}\AgdaSpace{}%
\AgdaSymbol{=}\AgdaSpace{}%
\AgdaFunction{renV}\AgdaSpace{}%
\AgdaInductiveConstructor{suc}\<%
\\
%
\\[\AgdaEmptyExtraSkip]%
\>[0]\AgdaFunction{coe}\AgdaSpace{}%
\AgdaSymbol{:}\AgdaSpace{}%
\AgdaSymbol{\{}\AgdaBound{t}\AgdaSpace{}%
\AgdaSymbol{:}\AgdaSpace{}%
\AgdaDatatype{Tm}\AgdaSpace{}%
\AgdaGeneralizable{Γ}\AgdaSpace{}%
\AgdaGeneralizable{ε}\AgdaSpace{}%
\AgdaGeneralizable{α}\AgdaSymbol{\}}\AgdaSpace{}%
\AgdaSymbol{→}\AgdaSpace{}%
\AgdaDatatype{Value}\AgdaSpace{}%
\AgdaBound{t}\AgdaSpace{}%
\AgdaSymbol{→}\AgdaSpace{}%
\AgdaDatatype{Tm}\AgdaSpace{}%
\AgdaGeneralizable{Γ}\AgdaSpace{}%
\AgdaGeneralizable{ε'}\AgdaSpace{}%
\AgdaGeneralizable{α}\<%
\\
\>[0]\AgdaFunction{coe}\AgdaSpace{}%
\AgdaSymbol{(}\AgdaOperator{\AgdaInductiveConstructor{ƛ}}\AgdaSpace{}%
\AgdaBound{t}\AgdaSymbol{)}\AgdaSpace{}%
\AgdaSymbol{=}\AgdaSpace{}%
\AgdaOperator{\AgdaInductiveConstructor{ƛ}}\AgdaSpace{}%
\AgdaBound{t}\<%
\\
\>[0]\AgdaFunction{coe}\AgdaSpace{}%
\AgdaInductiveConstructor{zero}\AgdaSpace{}%
\AgdaSymbol{=}\AgdaSpace{}%
\AgdaInductiveConstructor{zero}\<%
\\
\>[0]\AgdaFunction{coe}\AgdaSpace{}%
\AgdaSymbol{(}\AgdaInductiveConstructor{suc}\AgdaSpace{}%
\AgdaBound{v}\AgdaSymbol{)}\AgdaSpace{}%
\AgdaSymbol{=}\AgdaSpace{}%
\AgdaInductiveConstructor{suc}\AgdaSpace{}%
\AgdaSymbol{(}\AgdaFunction{coe}\AgdaSpace{}%
\AgdaBound{v}\AgdaSymbol{)}\<%
\end{code}

\begin{code}%
\>[0]\AgdaKeyword{data}\AgdaSpace{}%
\AgdaDatatype{PEC}\AgdaSpace{}%
\AgdaBound{Γ}\AgdaSpace{}%
\AgdaBound{ε}\AgdaSpace{}%
\AgdaBound{αₕ}\AgdaSpace{}%
\AgdaSymbol{:}\AgdaSpace{}%
\AgdaDatatype{Ty}\AgdaSpace{}%
\AgdaSymbol{→}\AgdaSpace{}%
\AgdaPrimitive{Set}\AgdaSpace{}%
\AgdaKeyword{where}\<%
\\
\>[0][@{}l@{\AgdaIndent{0}}]%
\>[2]\AgdaInductiveConstructor{⟨⟩}\AgdaSpace{}%
\AgdaSymbol{:}\AgdaSpace{}%
\AgdaDatatype{PEC}\AgdaSpace{}%
\AgdaBound{Γ}\AgdaSpace{}%
\AgdaBound{ε}\AgdaSpace{}%
\AgdaBound{αₕ}\AgdaSpace{}%
\AgdaBound{αₕ}\<%
\\
%
\>[2]\AgdaOperator{\AgdaInductiveConstructor{\AgdaUnderscore{}·₁\AgdaUnderscore{}}}\AgdaSpace{}%
\AgdaSymbol{:}\AgdaSpace{}%
\AgdaDatatype{PEC}\AgdaSpace{}%
\AgdaBound{Γ}\AgdaSpace{}%
\AgdaBound{ε}\AgdaSpace{}%
\AgdaBound{αₕ}\AgdaSpace{}%
\AgdaSymbol{(}\AgdaGeneralizable{α}\AgdaSpace{}%
\AgdaOperator{\AgdaInductiveConstructor{⇒}}\AgdaSpace{}%
\AgdaGeneralizable{β}\AgdaSpace{}%
\AgdaOperator{\AgdaInductiveConstructor{!}}\AgdaSpace{}%
\AgdaBound{ε}\AgdaSymbol{)}\AgdaSpace{}%
\AgdaSymbol{→}\AgdaSpace{}%
\AgdaDatatype{Tm}\AgdaSpace{}%
\AgdaBound{Γ}\AgdaSpace{}%
\AgdaBound{ε}\AgdaSpace{}%
\AgdaGeneralizable{α}\AgdaSpace{}%
\AgdaSymbol{→}\AgdaSpace{}%
\AgdaDatatype{PEC}\AgdaSpace{}%
\AgdaBound{Γ}\AgdaSpace{}%
\AgdaBound{ε}\AgdaSpace{}%
\AgdaBound{αₕ}\AgdaSpace{}%
\AgdaGeneralizable{β}\<%
\\
%
\>[2]\AgdaOperator{\AgdaInductiveConstructor{\AgdaUnderscore{}·₂\AgdaUnderscore{}}}\AgdaSpace{}%
\AgdaSymbol{:}\AgdaSpace{}%
\AgdaSymbol{\{}\AgdaBound{v}\AgdaSpace{}%
\AgdaSymbol{:}\AgdaSpace{}%
\AgdaDatatype{Tm}\AgdaSpace{}%
\AgdaBound{Γ}\AgdaSpace{}%
\AgdaBound{ε}\AgdaSpace{}%
\AgdaSymbol{(}\AgdaGeneralizable{α}\AgdaSpace{}%
\AgdaOperator{\AgdaInductiveConstructor{⇒}}\AgdaSpace{}%
\AgdaGeneralizable{β}\AgdaSpace{}%
\AgdaOperator{\AgdaInductiveConstructor{!}}\AgdaSpace{}%
\AgdaBound{ε}\AgdaSymbol{)\}}\AgdaSpace{}%
\AgdaSymbol{→}\AgdaSpace{}%
\AgdaDatatype{Value}\AgdaSpace{}%
\AgdaBound{v}\AgdaSpace{}%
\AgdaSymbol{→}\AgdaSpace{}%
\AgdaDatatype{PEC}\AgdaSpace{}%
\AgdaBound{Γ}\AgdaSpace{}%
\AgdaBound{ε}\AgdaSpace{}%
\AgdaBound{αₕ}\AgdaSpace{}%
\AgdaGeneralizable{α}\AgdaSpace{}%
\AgdaSymbol{→}\AgdaSpace{}%
\AgdaDatatype{PEC}\AgdaSpace{}%
\AgdaBound{Γ}\AgdaSpace{}%
\AgdaBound{ε}\AgdaSpace{}%
\AgdaBound{αₕ}\AgdaSpace{}%
\AgdaGeneralizable{β}\<%
\\
%
\>[2]\AgdaInductiveConstructor{suc}\AgdaSpace{}%
\AgdaSymbol{:}\AgdaSpace{}%
\AgdaDatatype{PEC}\AgdaSpace{}%
\AgdaBound{Γ}\AgdaSpace{}%
\AgdaBound{ε}\AgdaSpace{}%
\AgdaBound{αₕ}\AgdaSpace{}%
\AgdaInductiveConstructor{`ℕ}\AgdaSpace{}%
\AgdaSymbol{→}\AgdaSpace{}%
\AgdaDatatype{PEC}\AgdaSpace{}%
\AgdaBound{Γ}\AgdaSpace{}%
\AgdaBound{ε}\AgdaSpace{}%
\AgdaBound{αₕ}\AgdaSpace{}%
\AgdaInductiveConstructor{`ℕ}\<%
\\
%
\>[2]\AgdaInductiveConstructor{case}\AgdaSpace{}%
\AgdaSymbol{:}\AgdaSpace{}%
\AgdaDatatype{PEC}\AgdaSpace{}%
\AgdaBound{Γ}\AgdaSpace{}%
\AgdaBound{ε}\AgdaSpace{}%
\AgdaBound{αₕ}\AgdaSpace{}%
\AgdaInductiveConstructor{`ℕ}\AgdaSpace{}%
\AgdaSymbol{→}\AgdaSpace{}%
\AgdaDatatype{Tm}\AgdaSpace{}%
\AgdaBound{Γ}\AgdaSpace{}%
\AgdaBound{ε}\AgdaSpace{}%
\AgdaGeneralizable{α}\AgdaSpace{}%
\AgdaSymbol{→}\AgdaSpace{}%
\AgdaDatatype{Tm}\AgdaSpace{}%
\AgdaSymbol{(}\AgdaBound{Γ}\AgdaSpace{}%
\AgdaOperator{\AgdaInductiveConstructor{,}}\AgdaSpace{}%
\AgdaInductiveConstructor{`ℕ}\AgdaSymbol{)}\AgdaSpace{}%
\AgdaBound{ε}\AgdaSpace{}%
\AgdaGeneralizable{α}\AgdaSpace{}%
\AgdaSymbol{→}\AgdaSpace{}%
\AgdaDatatype{PEC}\AgdaSpace{}%
\AgdaBound{Γ}\AgdaSpace{}%
\AgdaBound{ε}\AgdaSpace{}%
\AgdaBound{αₕ}\AgdaSpace{}%
\AgdaGeneralizable{α}\<%
\\
%
\\[\AgdaEmptyExtraSkip]%
\>[0]\AgdaOperator{\AgdaFunction{\AgdaUnderscore{}⟨\AgdaUnderscore{}⟩}}\AgdaSpace{}%
\AgdaSymbol{:}\AgdaSpace{}%
\AgdaDatatype{PEC}\AgdaSpace{}%
\AgdaGeneralizable{Γ}\AgdaSpace{}%
\AgdaGeneralizable{ε}\AgdaSpace{}%
\AgdaGeneralizable{αₕ}\AgdaSpace{}%
\AgdaGeneralizable{α}\AgdaSpace{}%
\AgdaSymbol{→}\AgdaSpace{}%
\AgdaDatatype{Tm}\AgdaSpace{}%
\AgdaGeneralizable{Γ}\AgdaSpace{}%
\AgdaGeneralizable{ε}\AgdaSpace{}%
\AgdaGeneralizable{αₕ}\AgdaSpace{}%
\AgdaSymbol{→}\AgdaSpace{}%
\AgdaDatatype{Tm}\AgdaSpace{}%
\AgdaGeneralizable{Γ}\AgdaSpace{}%
\AgdaGeneralizable{ε}\AgdaSpace{}%
\AgdaGeneralizable{α}\<%
\\
\>[0]\AgdaInductiveConstructor{⟨⟩}\AgdaSpace{}%
\AgdaOperator{\AgdaFunction{⟨}}\AgdaSpace{}%
\AgdaBound{h}\AgdaSpace{}%
\AgdaOperator{\AgdaFunction{⟩}}\AgdaSpace{}%
\AgdaSymbol{=}\AgdaSpace{}%
\AgdaBound{h}\<%
\\
\>[0]\AgdaSymbol{(}\AgdaBound{c}\AgdaSpace{}%
\AgdaOperator{\AgdaInductiveConstructor{·₁}}\AgdaSpace{}%
\AgdaBound{t}\AgdaSymbol{)}\AgdaSpace{}%
\AgdaOperator{\AgdaFunction{⟨}}\AgdaSpace{}%
\AgdaBound{h}\AgdaSpace{}%
\AgdaOperator{\AgdaFunction{⟩}}\AgdaSpace{}%
\AgdaSymbol{=}\AgdaSpace{}%
\AgdaSymbol{(}\AgdaBound{c}\AgdaSpace{}%
\AgdaOperator{\AgdaFunction{⟨}}\AgdaSpace{}%
\AgdaBound{h}\AgdaSpace{}%
\AgdaOperator{\AgdaFunction{⟩}}\AgdaSymbol{)}\AgdaSpace{}%
\AgdaOperator{\AgdaInductiveConstructor{·}}\AgdaSpace{}%
\AgdaBound{t}\<%
\\
\>[0]\AgdaSymbol{(}\AgdaOperator{\AgdaInductiveConstructor{\AgdaUnderscore{}·₂\AgdaUnderscore{}}}\AgdaSpace{}%
\AgdaSymbol{\{}\AgdaArgument{v}\AgdaSpace{}%
\AgdaSymbol{=}\AgdaSpace{}%
\AgdaBound{t}\AgdaSymbol{\}}\AgdaSpace{}%
\AgdaSymbol{\AgdaUnderscore{}}\AgdaSpace{}%
\AgdaBound{c}\AgdaSymbol{)}\AgdaSpace{}%
\AgdaOperator{\AgdaFunction{⟨}}\AgdaSpace{}%
\AgdaBound{h}\AgdaSpace{}%
\AgdaOperator{\AgdaFunction{⟩}}\AgdaSpace{}%
\AgdaSymbol{=}\AgdaSpace{}%
\AgdaBound{t}\AgdaSpace{}%
\AgdaOperator{\AgdaInductiveConstructor{·}}\AgdaSpace{}%
\AgdaSymbol{(}\AgdaBound{c}\AgdaSpace{}%
\AgdaOperator{\AgdaFunction{⟨}}\AgdaSpace{}%
\AgdaBound{h}\AgdaSpace{}%
\AgdaOperator{\AgdaFunction{⟩}}\AgdaSymbol{)}\<%
\\
\>[0]\AgdaInductiveConstructor{suc}\AgdaSpace{}%
\AgdaBound{c}\AgdaSpace{}%
\AgdaOperator{\AgdaFunction{⟨}}\AgdaSpace{}%
\AgdaBound{h}\AgdaSpace{}%
\AgdaOperator{\AgdaFunction{⟩}}\AgdaSpace{}%
\AgdaSymbol{=}\AgdaSpace{}%
\AgdaInductiveConstructor{suc}\AgdaSpace{}%
\AgdaSymbol{(}\AgdaBound{c}\AgdaSpace{}%
\AgdaOperator{\AgdaFunction{⟨}}\AgdaSpace{}%
\AgdaBound{h}\AgdaSpace{}%
\AgdaOperator{\AgdaFunction{⟩}}\AgdaSymbol{)}\<%
\\
\>[0]\AgdaInductiveConstructor{case}\AgdaSpace{}%
\AgdaBound{c}\AgdaSpace{}%
\AgdaBound{z}\AgdaSpace{}%
\AgdaBound{s}\AgdaSpace{}%
\AgdaOperator{\AgdaFunction{⟨}}\AgdaSpace{}%
\AgdaBound{h}\AgdaSpace{}%
\AgdaOperator{\AgdaFunction{⟩}}\AgdaSpace{}%
\AgdaSymbol{=}\AgdaSpace{}%
\AgdaInductiveConstructor{case}\AgdaSpace{}%
\AgdaSymbol{(}\AgdaBound{c}\AgdaSpace{}%
\AgdaOperator{\AgdaFunction{⟨}}\AgdaSpace{}%
\AgdaBound{h}\AgdaSpace{}%
\AgdaOperator{\AgdaFunction{⟩}}\AgdaSymbol{)}\AgdaSpace{}%
\AgdaBound{z}\AgdaSpace{}%
\AgdaBound{s}\<%
\\
%
\\[\AgdaEmptyExtraSkip]%
\>[0]\AgdaFunction{renPEC}\AgdaSpace{}%
\AgdaSymbol{:}\AgdaSpace{}%
\AgdaFunction{Ren}\AgdaSpace{}%
\AgdaGeneralizable{Γ}\AgdaSpace{}%
\AgdaGeneralizable{Δ}\AgdaSpace{}%
\AgdaSymbol{→}\AgdaSpace{}%
\AgdaDatatype{PEC}\AgdaSpace{}%
\AgdaGeneralizable{Γ}\AgdaSpace{}%
\AgdaGeneralizable{ε}\AgdaSpace{}%
\AgdaGeneralizable{αₕ}\AgdaSpace{}%
\AgdaGeneralizable{α}\AgdaSpace{}%
\AgdaSymbol{→}\AgdaSpace{}%
\AgdaDatatype{PEC}\AgdaSpace{}%
\AgdaGeneralizable{Δ}\AgdaSpace{}%
\AgdaGeneralizable{ε}\AgdaSpace{}%
\AgdaGeneralizable{αₕ}\AgdaSpace{}%
\AgdaGeneralizable{α}\<%
\\
\>[0]\AgdaFunction{renPEC}\AgdaSpace{}%
\AgdaBound{ρ}\AgdaSpace{}%
\AgdaInductiveConstructor{⟨⟩}\AgdaSpace{}%
\AgdaSymbol{=}\AgdaSpace{}%
\AgdaInductiveConstructor{⟨⟩}\<%
\\
\>[0]\AgdaFunction{renPEC}\AgdaSpace{}%
\AgdaBound{ρ}\AgdaSpace{}%
\AgdaSymbol{(}\AgdaBound{c}\AgdaSpace{}%
\AgdaOperator{\AgdaInductiveConstructor{·₁}}\AgdaSpace{}%
\AgdaBound{t}\AgdaSymbol{)}\AgdaSpace{}%
\AgdaSymbol{=}\AgdaSpace{}%
\AgdaFunction{renPEC}\AgdaSpace{}%
\AgdaBound{ρ}\AgdaSpace{}%
\AgdaBound{c}\AgdaSpace{}%
\AgdaOperator{\AgdaInductiveConstructor{·₁}}\AgdaSpace{}%
\AgdaFunction{ren}\AgdaSpace{}%
\AgdaBound{ρ}\AgdaSpace{}%
\AgdaBound{t}\<%
\\
\>[0]\AgdaFunction{renPEC}\AgdaSpace{}%
\AgdaBound{ρ}\AgdaSpace{}%
\AgdaSymbol{(}\AgdaBound{v}\AgdaSpace{}%
\AgdaOperator{\AgdaInductiveConstructor{·₂}}\AgdaSpace{}%
\AgdaBound{c}\AgdaSymbol{)}\AgdaSpace{}%
\AgdaSymbol{=}\AgdaSpace{}%
\AgdaFunction{renV}\AgdaSpace{}%
\AgdaBound{ρ}\AgdaSpace{}%
\AgdaBound{v}\AgdaSpace{}%
\AgdaOperator{\AgdaInductiveConstructor{·₂}}\AgdaSpace{}%
\AgdaFunction{renPEC}\AgdaSpace{}%
\AgdaBound{ρ}\AgdaSpace{}%
\AgdaBound{c}\<%
\\
\>[0]\AgdaFunction{renPEC}\AgdaSpace{}%
\AgdaBound{ρ}\AgdaSpace{}%
\AgdaSymbol{(}\AgdaInductiveConstructor{suc}\AgdaSpace{}%
\AgdaBound{c}\AgdaSymbol{)}\AgdaSpace{}%
\AgdaSymbol{=}\AgdaSpace{}%
\AgdaInductiveConstructor{suc}\AgdaSpace{}%
\AgdaSymbol{(}\AgdaFunction{renPEC}\AgdaSpace{}%
\AgdaBound{ρ}\AgdaSpace{}%
\AgdaBound{c}\AgdaSymbol{)}\<%
\\
\>[0]\AgdaFunction{renPEC}\AgdaSpace{}%
\AgdaBound{ρ}\AgdaSpace{}%
\AgdaSymbol{(}\AgdaInductiveConstructor{case}\AgdaSpace{}%
\AgdaBound{c}\AgdaSpace{}%
\AgdaBound{z}\AgdaSpace{}%
\AgdaBound{s}\AgdaSymbol{)}\AgdaSpace{}%
\AgdaSymbol{=}\AgdaSpace{}%
\AgdaInductiveConstructor{case}\AgdaSpace{}%
\AgdaSymbol{(}\AgdaFunction{renPEC}\AgdaSpace{}%
\AgdaBound{ρ}\AgdaSpace{}%
\AgdaBound{c}\AgdaSymbol{)}\AgdaSpace{}%
\AgdaSymbol{(}\AgdaFunction{ren}\AgdaSpace{}%
\AgdaBound{ρ}\AgdaSpace{}%
\AgdaBound{z}\AgdaSymbol{)}\AgdaSpace{}%
\AgdaSymbol{(}\AgdaFunction{ren}\AgdaSpace{}%
\AgdaSymbol{(}\AgdaFunction{ext}\AgdaSpace{}%
\AgdaBound{ρ}\AgdaSymbol{)}\AgdaSpace{}%
\AgdaBound{s}\AgdaSymbol{)}\<%
\\
%
\\[\AgdaEmptyExtraSkip]%
\>[0]\AgdaFunction{↑PEC}\AgdaSpace{}%
\AgdaSymbol{:}\AgdaSpace{}%
\AgdaDatatype{PEC}\AgdaSpace{}%
\AgdaGeneralizable{Γ}\AgdaSpace{}%
\AgdaGeneralizable{ε}\AgdaSpace{}%
\AgdaGeneralizable{αₕ}\AgdaSpace{}%
\AgdaGeneralizable{α}\AgdaSpace{}%
\AgdaSymbol{→}\AgdaSpace{}%
\AgdaDatatype{PEC}\AgdaSpace{}%
\AgdaSymbol{(}\AgdaGeneralizable{Γ}\AgdaSpace{}%
\AgdaOperator{\AgdaInductiveConstructor{,}}\AgdaSpace{}%
\AgdaGeneralizable{β}\AgdaSymbol{)}\AgdaSpace{}%
\AgdaGeneralizable{ε}\AgdaSpace{}%
\AgdaGeneralizable{αₕ}\AgdaSpace{}%
\AgdaGeneralizable{α}\<%
\\
\>[0]\AgdaFunction{↑PEC}\AgdaSpace{}%
\AgdaSymbol{=}\AgdaSpace{}%
\AgdaFunction{renPEC}\AgdaSpace{}%
\AgdaInductiveConstructor{suc}\<%
\end{code}

\subsection{Reduction rules}

\begin{AgdaAlign}
\begin{code}%
\>[0]\AgdaKeyword{data}\AgdaSpace{}%
\AgdaOperator{\AgdaDatatype{\AgdaUnderscore{}⟶\AgdaUnderscore{}}}\AgdaSpace{}%
\AgdaSymbol{:}\AgdaSpace{}%
\AgdaDatatype{Tm}\AgdaSpace{}%
\AgdaGeneralizable{Γ}\AgdaSpace{}%
\AgdaGeneralizable{ε}\AgdaSpace{}%
\AgdaGeneralizable{α}\AgdaSpace{}%
\AgdaSymbol{→}\AgdaSpace{}%
\AgdaDatatype{Tm}\AgdaSpace{}%
\AgdaGeneralizable{Γ}\AgdaSpace{}%
\AgdaGeneralizable{ε}\AgdaSpace{}%
\AgdaGeneralizable{α}\AgdaSpace{}%
\AgdaSymbol{→}\AgdaSpace{}%
\AgdaPrimitive{Set}\AgdaSpace{}%
\AgdaKeyword{where}\<%
\\
\>[0][@{}l@{\AgdaIndent{0}}]%
\>[2]\AgdaInductiveConstructor{app}\AgdaSpace{}%
\AgdaSymbol{:}\AgdaSpace{}%
\AgdaSymbol{∀}\AgdaSpace{}%
\AgdaSymbol{\{}\AgdaBound{t}\AgdaSpace{}%
\AgdaSymbol{:}\AgdaSpace{}%
\AgdaDatatype{Tm}\AgdaSpace{}%
\AgdaSymbol{(}\AgdaGeneralizable{Γ}\AgdaSpace{}%
\AgdaOperator{\AgdaInductiveConstructor{,}}\AgdaSpace{}%
\AgdaGeneralizable{α}\AgdaSymbol{)}\AgdaSpace{}%
\AgdaGeneralizable{ε}\AgdaSpace{}%
\AgdaGeneralizable{β}\AgdaSymbol{\}}\AgdaSpace{}%
\AgdaSymbol{\{}\AgdaBound{u}\AgdaSymbol{\}}\<%
\\
\>[2][@{}l@{\AgdaIndent{0}}]%
\>[4]\AgdaSymbol{→}\AgdaSpace{}%
\AgdaSymbol{(}\AgdaBound{v}\AgdaSpace{}%
\AgdaSymbol{:}\AgdaSpace{}%
\AgdaDatatype{Value}\AgdaSpace{}%
\AgdaBound{u}\AgdaSymbol{)}\<%
\\
%
\>[4]\AgdaSymbol{→}\AgdaSpace{}%
\AgdaSymbol{(}\AgdaOperator{\AgdaInductiveConstructor{ƛ}}\AgdaSpace{}%
\AgdaBound{t}\AgdaSymbol{)}\AgdaSpace{}%
\AgdaOperator{\AgdaInductiveConstructor{·}}\AgdaSpace{}%
\AgdaBound{u}\AgdaSpace{}%
\AgdaOperator{\AgdaDatatype{⟶}}\AgdaSpace{}%
\AgdaBound{t}\AgdaSpace{}%
\AgdaOperator{\AgdaFunction{[}}\AgdaSpace{}%
\AgdaFunction{coe}\AgdaSpace{}%
\AgdaBound{v}\AgdaSpace{}%
\AgdaOperator{\AgdaFunction{]}}\<%
\\
%
\\[\AgdaEmptyExtraSkip]%
%
\>[2]\AgdaInductiveConstructor{case-zero}\AgdaSpace{}%
\AgdaSymbol{:}\AgdaSpace{}%
\AgdaSymbol{∀}\AgdaSpace{}%
\AgdaSymbol{\{}\AgdaBound{z}\AgdaSpace{}%
\AgdaSymbol{:}\AgdaSpace{}%
\AgdaDatatype{Tm}\AgdaSpace{}%
\AgdaGeneralizable{Γ}\AgdaSpace{}%
\AgdaGeneralizable{ε}\AgdaSpace{}%
\AgdaGeneralizable{α}\AgdaSymbol{\}}\AgdaSpace{}%
\AgdaSymbol{\{}\AgdaBound{s}\AgdaSymbol{\}}\<%
\\
\>[2][@{}l@{\AgdaIndent{0}}]%
\>[4]\AgdaSymbol{→}\AgdaSpace{}%
\AgdaInductiveConstructor{case}\AgdaSpace{}%
\AgdaInductiveConstructor{zero}\AgdaSpace{}%
\AgdaBound{z}\AgdaSpace{}%
\AgdaBound{s}\AgdaSpace{}%
\AgdaOperator{\AgdaDatatype{⟶}}\AgdaSpace{}%
\AgdaBound{z}\<%
\\
%
\\[\AgdaEmptyExtraSkip]%
%
\>[2]\AgdaInductiveConstructor{case-suc}\AgdaSpace{}%
\AgdaSymbol{:}\AgdaSpace{}%
\AgdaSymbol{∀}\AgdaSpace{}%
\AgdaSymbol{\{}\AgdaBound{z}\AgdaSpace{}%
\AgdaSymbol{:}\AgdaSpace{}%
\AgdaDatatype{Tm}\AgdaSpace{}%
\AgdaGeneralizable{Γ}\AgdaSpace{}%
\AgdaGeneralizable{ε}\AgdaSpace{}%
\AgdaGeneralizable{α}\AgdaSymbol{\}}\AgdaSpace{}%
\AgdaSymbol{\{}\AgdaBound{s}\AgdaSpace{}%
\AgdaBound{n}\AgdaSymbol{\}}\<%
\\
\>[2][@{}l@{\AgdaIndent{0}}]%
\>[4]\AgdaSymbol{→}\AgdaSpace{}%
\AgdaSymbol{(}\AgdaBound{v}\AgdaSpace{}%
\AgdaSymbol{:}\AgdaSpace{}%
\AgdaDatatype{Value}\AgdaSpace{}%
\AgdaBound{n}\AgdaSymbol{)}\<%
\\
%
\>[4]\AgdaSymbol{→}\AgdaSpace{}%
\AgdaInductiveConstructor{case}\AgdaSpace{}%
\AgdaSymbol{(}\AgdaInductiveConstructor{suc}\AgdaSpace{}%
\AgdaBound{n}\AgdaSymbol{)}\AgdaSpace{}%
\AgdaBound{z}\AgdaSpace{}%
\AgdaBound{s}\AgdaSpace{}%
\AgdaOperator{\AgdaDatatype{⟶}}\AgdaSpace{}%
\AgdaBound{s}\AgdaSpace{}%
\AgdaOperator{\AgdaFunction{[}}\AgdaSpace{}%
\AgdaFunction{coe}\AgdaSpace{}%
\AgdaBound{v}\AgdaSpace{}%
\AgdaOperator{\AgdaFunction{]}}\<%
\\
%
\\[\AgdaEmptyExtraSkip]%
%
\>[2]\AgdaInductiveConstructor{delimit-value}\AgdaSpace{}%
\AgdaSymbol{:}\AgdaSpace{}%
\AgdaSymbol{\{}\AgdaBound{t}\AgdaSpace{}%
\AgdaSymbol{:}\AgdaSpace{}%
\AgdaDatatype{Tm}\AgdaSpace{}%
\AgdaGeneralizable{Γ}\AgdaSpace{}%
\AgdaSymbol{(}\AgdaGeneralizable{ε}\AgdaSpace{}%
\AgdaOperator{\AgdaInductiveConstructor{,}}\AgdaSpace{}%
\AgdaGeneralizable{α}\AgdaSymbol{)}\AgdaSpace{}%
\AgdaGeneralizable{α}\AgdaSymbol{\}}\<%
\\
\>[2][@{}l@{\AgdaIndent{0}}]%
\>[4]\AgdaSymbol{→}\AgdaSpace{}%
\AgdaSymbol{(}\AgdaBound{v}\AgdaSpace{}%
\AgdaSymbol{:}\AgdaSpace{}%
\AgdaDatatype{Value}\AgdaSpace{}%
\AgdaBound{t}\AgdaSymbol{)}\<%
\\
%
\>[4]\AgdaSymbol{→}\AgdaSpace{}%
\AgdaInductiveConstructor{delimit}\AgdaSpace{}%
\AgdaBound{t}\AgdaSpace{}%
\AgdaOperator{\AgdaDatatype{⟶}}\AgdaSpace{}%
\AgdaFunction{coe}\AgdaSpace{}%
\AgdaBound{v}\<%
\\
%
\\[\AgdaEmptyExtraSkip]%
%
\>[2]\AgdaInductiveConstructor{delimit-grab}\AgdaSpace{}%
\AgdaSymbol{:}\AgdaSpace{}%
\AgdaSymbol{∀}\AgdaSpace{}%
\AgdaSymbol{\{}\AgdaBound{t}\AgdaSpace{}%
\AgdaSymbol{:}\AgdaSpace{}%
\AgdaDatatype{Tm}\AgdaSpace{}%
\AgdaGeneralizable{Γ}\AgdaSpace{}%
\AgdaSymbol{(}\AgdaGeneralizable{ε}\AgdaSpace{}%
\AgdaOperator{\AgdaInductiveConstructor{,}}\AgdaSpace{}%
\AgdaGeneralizable{α}\AgdaSymbol{)}\AgdaSpace{}%
\AgdaGeneralizable{α}\AgdaSymbol{\}}\AgdaSpace{}%
\AgdaSymbol{(}\AgdaBound{c}\AgdaSpace{}%
\AgdaSymbol{:}\AgdaSpace{}%
\AgdaDatatype{PEC}\AgdaSpace{}%
\AgdaGeneralizable{Γ}\AgdaSpace{}%
\AgdaSymbol{(}\AgdaGeneralizable{ε}\AgdaSpace{}%
\AgdaOperator{\AgdaInductiveConstructor{,}}\AgdaSpace{}%
\AgdaGeneralizable{α}\AgdaSymbol{)}\AgdaSpace{}%
\AgdaGeneralizable{αₕ}\AgdaSpace{}%
\AgdaGeneralizable{α}\AgdaSymbol{)}\AgdaSpace{}%
\AgdaSymbol{\{}\AgdaBound{u}\AgdaSymbol{\}}\<%
\\
\>[2][@{}l@{\AgdaIndent{0}}]%
\>[4]\AgdaSymbol{→}\AgdaSpace{}%
\AgdaBound{t}\AgdaSpace{}%
\AgdaOperator{\AgdaDatatype{≡}}\AgdaSpace{}%
\AgdaBound{c}\AgdaSpace{}%
\AgdaOperator{\AgdaFunction{⟨}}\AgdaSpace{}%
\AgdaInductiveConstructor{grab}\AgdaSpace{}%
\AgdaBound{u}\AgdaSpace{}%
\AgdaOperator{\AgdaFunction{⟩}}\<%
\\
%
\>[4]\AgdaSymbol{→}\AgdaSpace{}%
\AgdaInductiveConstructor{delimit}\AgdaSpace{}%
\AgdaBound{t}\AgdaSpace{}%
\AgdaOperator{\AgdaDatatype{⟶}}\AgdaSpace{}%
\AgdaBound{u}\AgdaSpace{}%
\AgdaOperator{\AgdaFunction{[}}\AgdaSpace{}%
\AgdaOperator{\AgdaInductiveConstructor{ƛ}}\AgdaSpace{}%
\AgdaInductiveConstructor{delimit}\AgdaSpace{}%
\AgdaSymbol{(}\AgdaFunction{↑PEC}\AgdaSpace{}%
\AgdaBound{c}\AgdaSpace{}%
\AgdaOperator{\AgdaFunction{⟨}}\AgdaSpace{}%
\AgdaFunction{var\#}\AgdaSpace{}%
\AgdaNumber{0}\AgdaSpace{}%
\AgdaOperator{\AgdaFunction{⟩}}\AgdaSymbol{)}\AgdaSpace{}%
\AgdaOperator{\AgdaFunction{]}}\<%
\\
%
\\[\AgdaEmptyExtraSkip]%
%
\>[2]\AgdaInductiveConstructor{unroll}\AgdaSpace{}%
\AgdaSymbol{:}\AgdaSpace{}%
\AgdaSymbol{\{}\AgdaBound{t}\AgdaSpace{}%
\AgdaSymbol{:}\AgdaSpace{}%
\AgdaDatatype{Tm}\AgdaSpace{}%
\AgdaSymbol{(}\AgdaGeneralizable{Γ}\AgdaSpace{}%
\AgdaOperator{\AgdaInductiveConstructor{,}}\AgdaSpace{}%
\AgdaGeneralizable{α}\AgdaSpace{}%
\AgdaOperator{\AgdaInductiveConstructor{⇒}}\AgdaSpace{}%
\AgdaGeneralizable{β}\AgdaSpace{}%
\AgdaOperator{\AgdaInductiveConstructor{!}}\AgdaSpace{}%
\AgdaGeneralizable{ε}\AgdaSymbol{)}\AgdaSpace{}%
\AgdaGeneralizable{ε}\AgdaSpace{}%
\AgdaSymbol{(}\AgdaGeneralizable{α}\AgdaSpace{}%
\AgdaOperator{\AgdaInductiveConstructor{⇒}}\AgdaSpace{}%
\AgdaGeneralizable{β}\AgdaSpace{}%
\AgdaOperator{\AgdaInductiveConstructor{!}}\AgdaSpace{}%
\AgdaGeneralizable{ε}\AgdaSymbol{)\}}\<%
\\
\>[2][@{}l@{\AgdaIndent{0}}]%
\>[4]\AgdaSymbol{→}\AgdaSpace{}%
\AgdaOperator{\AgdaInductiveConstructor{μ}}\AgdaSpace{}%
\AgdaBound{t}\AgdaSpace{}%
\AgdaOperator{\AgdaDatatype{⟶}}\AgdaSpace{}%
\AgdaBound{t}\AgdaSpace{}%
\AgdaOperator{\AgdaFunction{[}}\AgdaSpace{}%
\AgdaOperator{\AgdaInductiveConstructor{ƛ}}\AgdaSpace{}%
\AgdaSymbol{(}\AgdaOperator{\AgdaFunction{↑}}\AgdaSpace{}%
\AgdaSymbol{(}\AgdaOperator{\AgdaInductiveConstructor{μ}}\AgdaSpace{}%
\AgdaBound{t}\AgdaSymbol{))}\AgdaSpace{}%
\AgdaOperator{\AgdaInductiveConstructor{·}}\AgdaSpace{}%
\AgdaFunction{var\#}\AgdaSpace{}%
\AgdaNumber{0}\AgdaSpace{}%
\AgdaOperator{\AgdaFunction{]}}\<%
\end{code}

Here are the congruence rules.

\begin{code}%
%
\>[2]\AgdaInductiveConstructor{cong-·₁}\AgdaSpace{}%
\AgdaSymbol{:}\AgdaSpace{}%
\AgdaSymbol{∀}\AgdaSpace{}%
\AgdaSymbol{\{}\AgdaBound{t}\AgdaSpace{}%
\AgdaBound{t'}\AgdaSpace{}%
\AgdaSymbol{:}\AgdaSpace{}%
\AgdaDatatype{Tm}\AgdaSpace{}%
\AgdaGeneralizable{Γ}\AgdaSpace{}%
\AgdaGeneralizable{ε}\AgdaSpace{}%
\AgdaSymbol{(}\AgdaGeneralizable{α}\AgdaSpace{}%
\AgdaOperator{\AgdaInductiveConstructor{⇒}}\AgdaSpace{}%
\AgdaGeneralizable{β}\AgdaSpace{}%
\AgdaOperator{\AgdaInductiveConstructor{!}}\AgdaSpace{}%
\AgdaGeneralizable{ε}\AgdaSymbol{)\}}\AgdaSpace{}%
\AgdaSymbol{\{}\AgdaBound{u}\AgdaSymbol{\}}\<%
\\
\>[2][@{}l@{\AgdaIndent{0}}]%
\>[4]\AgdaSymbol{→}\AgdaSpace{}%
\AgdaBound{t}\AgdaSpace{}%
\AgdaOperator{\AgdaDatatype{⟶}}\AgdaSpace{}%
\AgdaBound{t'}\<%
\\
%
\>[4]\AgdaSymbol{→}\AgdaSpace{}%
\AgdaBound{t}\AgdaSpace{}%
\AgdaOperator{\AgdaInductiveConstructor{·}}\AgdaSpace{}%
\AgdaBound{u}\AgdaSpace{}%
\AgdaOperator{\AgdaDatatype{⟶}}\AgdaSpace{}%
\AgdaBound{t'}\AgdaSpace{}%
\AgdaOperator{\AgdaInductiveConstructor{·}}\AgdaSpace{}%
\AgdaBound{u}\<%
\\
%
\\[\AgdaEmptyExtraSkip]%
%
\>[2]\AgdaInductiveConstructor{cong-·₂}\AgdaSpace{}%
\AgdaSymbol{:}\AgdaSpace{}%
\AgdaSymbol{∀}\AgdaSpace{}%
\AgdaSymbol{\{}\AgdaBound{v}\AgdaSpace{}%
\AgdaSymbol{:}\AgdaSpace{}%
\AgdaDatatype{Tm}\AgdaSpace{}%
\AgdaGeneralizable{Γ}\AgdaSpace{}%
\AgdaGeneralizable{ε}\AgdaSpace{}%
\AgdaSymbol{(}\AgdaGeneralizable{α}\AgdaSpace{}%
\AgdaOperator{\AgdaInductiveConstructor{⇒}}\AgdaSpace{}%
\AgdaGeneralizable{β}\AgdaSpace{}%
\AgdaOperator{\AgdaInductiveConstructor{!}}\AgdaSpace{}%
\AgdaGeneralizable{ε}\AgdaSymbol{)\}}\AgdaSpace{}%
\AgdaSymbol{\{}\AgdaBound{t}\AgdaSpace{}%
\AgdaBound{t'}\AgdaSymbol{\}}\<%
\\
\>[2][@{}l@{\AgdaIndent{0}}]%
\>[4]\AgdaSymbol{→}\AgdaSpace{}%
\AgdaDatatype{Value}\AgdaSpace{}%
\AgdaBound{v}\<%
\\
%
\>[4]\AgdaSymbol{→}\AgdaSpace{}%
\AgdaBound{t}\AgdaSpace{}%
\AgdaOperator{\AgdaDatatype{⟶}}\AgdaSpace{}%
\AgdaBound{t'}\<%
\\
%
\>[4]\AgdaSymbol{→}\AgdaSpace{}%
\AgdaBound{v}\AgdaSpace{}%
\AgdaOperator{\AgdaInductiveConstructor{·}}\AgdaSpace{}%
\AgdaBound{t}\AgdaSpace{}%
\AgdaOperator{\AgdaDatatype{⟶}}\AgdaSpace{}%
\AgdaBound{v}\AgdaSpace{}%
\AgdaOperator{\AgdaInductiveConstructor{·}}\AgdaSpace{}%
\AgdaBound{t'}\<%
\\
%
\\[\AgdaEmptyExtraSkip]%
%
\>[2]\AgdaInductiveConstructor{cong-delimit}\AgdaSpace{}%
\AgdaSymbol{:}\AgdaSpace{}%
\AgdaSymbol{\{}\AgdaBound{t}\AgdaSpace{}%
\AgdaBound{t'}\AgdaSpace{}%
\AgdaSymbol{:}\AgdaSpace{}%
\AgdaDatatype{Tm}\AgdaSpace{}%
\AgdaGeneralizable{Γ}\AgdaSpace{}%
\AgdaSymbol{(}\AgdaGeneralizable{ε}\AgdaSpace{}%
\AgdaOperator{\AgdaInductiveConstructor{,}}\AgdaSpace{}%
\AgdaGeneralizable{α}\AgdaSymbol{)}\AgdaSpace{}%
\AgdaGeneralizable{α}\AgdaSymbol{\}}\<%
\\
\>[2][@{}l@{\AgdaIndent{0}}]%
\>[4]\AgdaSymbol{→}\AgdaSpace{}%
\AgdaBound{t}\AgdaSpace{}%
\AgdaOperator{\AgdaDatatype{⟶}}\AgdaSpace{}%
\AgdaBound{t'}\<%
\\
%
\>[4]\AgdaSymbol{→}\AgdaSpace{}%
\AgdaInductiveConstructor{delimit}\AgdaSpace{}%
\AgdaBound{t}\AgdaSpace{}%
\AgdaOperator{\AgdaDatatype{⟶}}\AgdaSpace{}%
\AgdaInductiveConstructor{delimit}\AgdaSpace{}%
\AgdaBound{t'}\<%
\\
%
\\[\AgdaEmptyExtraSkip]%
%
\>[2]\AgdaInductiveConstructor{cong-suc}\AgdaSpace{}%
\AgdaSymbol{:}\AgdaSpace{}%
\AgdaSymbol{\{}\AgdaBound{t}\AgdaSpace{}%
\AgdaBound{t'}\AgdaSpace{}%
\AgdaSymbol{:}\AgdaSpace{}%
\AgdaDatatype{Tm}\AgdaSpace{}%
\AgdaGeneralizable{Γ}\AgdaSpace{}%
\AgdaGeneralizable{ε}\AgdaSpace{}%
\AgdaInductiveConstructor{`ℕ}\AgdaSymbol{\}}\<%
\\
\>[2][@{}l@{\AgdaIndent{0}}]%
\>[4]\AgdaSymbol{→}\AgdaSpace{}%
\AgdaBound{t}\AgdaSpace{}%
\AgdaOperator{\AgdaDatatype{⟶}}\AgdaSpace{}%
\AgdaBound{t'}\<%
\\
%
\>[4]\AgdaSymbol{→}\AgdaSpace{}%
\AgdaInductiveConstructor{suc}\AgdaSpace{}%
\AgdaBound{t}\AgdaSpace{}%
\AgdaOperator{\AgdaDatatype{⟶}}\AgdaSpace{}%
\AgdaInductiveConstructor{suc}\AgdaSpace{}%
\AgdaBound{t'}\<%
\\
%
\\[\AgdaEmptyExtraSkip]%
%
\>[2]\AgdaInductiveConstructor{cong-case}\AgdaSpace{}%
\AgdaSymbol{:}\AgdaSpace{}%
\AgdaSymbol{∀}\AgdaSpace{}%
\AgdaSymbol{\{}\AgdaBound{z}\AgdaSpace{}%
\AgdaSymbol{:}\AgdaSpace{}%
\AgdaDatatype{Tm}\AgdaSpace{}%
\AgdaGeneralizable{Γ}\AgdaSpace{}%
\AgdaGeneralizable{ε}\AgdaSpace{}%
\AgdaGeneralizable{α}\AgdaSymbol{\}}\AgdaSpace{}%
\AgdaSymbol{\{}\AgdaBound{s}\AgdaSpace{}%
\AgdaBound{n}\AgdaSpace{}%
\AgdaBound{n'}\AgdaSymbol{\}}\<%
\\
\>[2][@{}l@{\AgdaIndent{0}}]%
\>[4]\AgdaSymbol{→}\AgdaSpace{}%
\AgdaBound{n}\AgdaSpace{}%
\AgdaOperator{\AgdaDatatype{⟶}}\AgdaSpace{}%
\AgdaBound{n'}\<%
\\
%
\>[4]\AgdaSymbol{→}\AgdaSpace{}%
\AgdaInductiveConstructor{case}\AgdaSpace{}%
\AgdaBound{n}\AgdaSpace{}%
\AgdaBound{z}\AgdaSpace{}%
\AgdaBound{s}\AgdaSpace{}%
\AgdaOperator{\AgdaDatatype{⟶}}\AgdaSpace{}%
\AgdaInductiveConstructor{case}\AgdaSpace{}%
\AgdaBound{n'}\AgdaSpace{}%
\AgdaBound{z}\AgdaSpace{}%
\AgdaBound{s}\<%
\end{code}
\end{AgdaAlign}

\begin{code}%
\>[0]\AgdaKeyword{open}\AgdaSpace{}%
\AgdaKeyword{import}\AgdaSpace{}%
\AgdaModule{Relation.Binary.Construct.Closure.ReflexiveTransitive}\AgdaSpace{}%
\AgdaKeyword{using}\AgdaSpace{}%
\AgdaSymbol{(}\AgdaSpace{}%
\AgdaDatatype{Star}\AgdaSpace{}%
\AgdaSymbol{)}\<%
\\
\>[0]\AgdaKeyword{open}\AgdaSpace{}%
\AgdaKeyword{import}\AgdaSpace{}%
\AgdaModule{Relation.Binary.Construct.Closure.ReflexiveTransitive.Properties}\<%
\\
\>[0][@{}l@{\AgdaIndent{0}}]%
\>[2]\AgdaKeyword{using}\AgdaSpace{}%
\AgdaSymbol{(}\AgdaSpace{}%
\AgdaKeyword{module}\AgdaSpace{}%
\AgdaModule{StarReasoning}\AgdaSpace{}%
\AgdaSymbol{)}\<%
\\
%
\\[\AgdaEmptyExtraSkip]%
\>[0]\AgdaOperator{\AgdaFunction{\AgdaUnderscore{}⟶*\AgdaUnderscore{}}}\AgdaSpace{}%
\AgdaSymbol{:}\AgdaSpace{}%
\AgdaDatatype{Tm}\AgdaSpace{}%
\AgdaGeneralizable{Γ}\AgdaSpace{}%
\AgdaGeneralizable{ε}\AgdaSpace{}%
\AgdaGeneralizable{α}\AgdaSpace{}%
\AgdaSymbol{→}\AgdaSpace{}%
\AgdaDatatype{Tm}\AgdaSpace{}%
\AgdaGeneralizable{Γ}\AgdaSpace{}%
\AgdaGeneralizable{ε}\AgdaSpace{}%
\AgdaGeneralizable{α}\AgdaSpace{}%
\AgdaSymbol{→}\AgdaSpace{}%
\AgdaPrimitive{Set}\<%
\\
\>[0]\AgdaOperator{\AgdaFunction{\AgdaUnderscore{}⟶*\AgdaUnderscore{}}}\AgdaSpace{}%
\AgdaSymbol{=}\AgdaSpace{}%
\AgdaDatatype{Star}\AgdaSpace{}%
\AgdaOperator{\AgdaDatatype{\AgdaUnderscore{}⟶\AgdaUnderscore{}}}\<%
\\
%
\\[\AgdaEmptyExtraSkip]%
\>[0]\AgdaKeyword{module}\AgdaSpace{}%
\AgdaModule{⟶*-Reasoning}\AgdaSpace{}%
\AgdaSymbol{\{}\AgdaBound{Γ}\AgdaSpace{}%
\AgdaBound{ε}\AgdaSpace{}%
\AgdaBound{α}\AgdaSymbol{\}}\AgdaSpace{}%
\AgdaKeyword{where}\<%
\\
\>[0][@{}l@{\AgdaIndent{0}}]%
\>[2]\AgdaKeyword{open}\AgdaSpace{}%
\AgdaModule{StarReasoning}\AgdaSpace{}%
\AgdaSymbol{(}\AgdaOperator{\AgdaDatatype{\AgdaUnderscore{}⟶\AgdaUnderscore{}}}\AgdaSpace{}%
\AgdaSymbol{\{}\AgdaBound{Γ}\AgdaSymbol{\}}\AgdaSpace{}%
\AgdaSymbol{\{}\AgdaBound{ε}\AgdaSymbol{\}}\AgdaSpace{}%
\AgdaSymbol{\{}\AgdaBound{α}\AgdaSymbol{\})}\AgdaSpace{}%
\AgdaKeyword{public}\<%
\end{code}

\begin{code}%
\>[0]\AgdaComment{--\ delimit\ (suc\ (grab\ k.\ k\ (k\ 0)))}\<%
\\
\>[0]\AgdaFunction{ex}\AgdaSpace{}%
\AgdaSymbol{:}\AgdaSpace{}%
\AgdaDatatype{Tm}\AgdaSpace{}%
\AgdaInductiveConstructor{∙}\AgdaSpace{}%
\AgdaInductiveConstructor{ι}\AgdaSpace{}%
\AgdaInductiveConstructor{`ℕ}\<%
\\
\>[0]\AgdaFunction{ex}\AgdaSpace{}%
\AgdaSymbol{=}\AgdaSpace{}%
\AgdaInductiveConstructor{delimit}\AgdaSpace{}%
\AgdaSymbol{(}\AgdaInductiveConstructor{suc}\AgdaSpace{}%
\AgdaSymbol{(}\AgdaInductiveConstructor{grab}\AgdaSpace{}%
\AgdaSymbol{(}\AgdaFunction{var\#}\AgdaSpace{}%
\AgdaNumber{0}\AgdaSpace{}%
\AgdaOperator{\AgdaInductiveConstructor{·}}\AgdaSpace{}%
\AgdaSymbol{(}\AgdaFunction{var\#}\AgdaSpace{}%
\AgdaNumber{0}\AgdaSpace{}%
\AgdaOperator{\AgdaInductiveConstructor{·}}\AgdaSpace{}%
\AgdaInductiveConstructor{zero}\AgdaSymbol{))))}\<%
\\
%
\\[\AgdaEmptyExtraSkip]%
\>[0]\AgdaFunction{\AgdaUnderscore{}}\AgdaSpace{}%
\AgdaSymbol{:}\AgdaSpace{}%
\AgdaFunction{ex}\AgdaSpace{}%
\AgdaOperator{\AgdaFunction{⟶*}}\AgdaSpace{}%
\AgdaInductiveConstructor{suc}\AgdaSpace{}%
\AgdaSymbol{(}\AgdaInductiveConstructor{suc}\AgdaSpace{}%
\AgdaInductiveConstructor{zero}\AgdaSymbol{)}\<%
\\
\>[0]\AgdaSymbol{\AgdaUnderscore{}}%
\>[1440I]\AgdaSymbol{=}\<%
\\
\>[.][@{}l@{}]\<[1440I]%
\>[2]\AgdaOperator{\AgdaFunction{begin}}\<%
\\
\>[2][@{}l@{\AgdaIndent{0}}]%
\>[4]\AgdaFunction{ex}\<%
\\
%
\>[2]\AgdaFunction{≡⟨⟩}\<%
\\
\>[2][@{}l@{\AgdaIndent{0}}]%
\>[4]\AgdaInductiveConstructor{delimit}\AgdaSpace{}%
\AgdaSymbol{(}\AgdaInductiveConstructor{suc}\AgdaSpace{}%
\AgdaSymbol{(}\AgdaInductiveConstructor{grab}\AgdaSpace{}%
\AgdaSymbol{(}\AgdaFunction{var\#}\AgdaSpace{}%
\AgdaNumber{0}\AgdaSpace{}%
\AgdaOperator{\AgdaInductiveConstructor{·}}\AgdaSpace{}%
\AgdaSymbol{(}\AgdaFunction{var\#}\AgdaSpace{}%
\AgdaNumber{0}\AgdaSpace{}%
\AgdaOperator{\AgdaInductiveConstructor{·}}\AgdaSpace{}%
\AgdaInductiveConstructor{zero}\AgdaSymbol{))))}\<%
\\
%
\>[2]\AgdaFunction{⟶⟨}\AgdaSpace{}%
\AgdaInductiveConstructor{delimit-grab}\AgdaSpace{}%
\AgdaSymbol{(}\AgdaInductiveConstructor{suc}\AgdaSpace{}%
\AgdaInductiveConstructor{⟨⟩}\AgdaSymbol{)}\AgdaSpace{}%
\AgdaInductiveConstructor{refl}\AgdaSpace{}%
\AgdaFunction{⟩}\<%
\\
\>[2][@{}l@{\AgdaIndent{0}}]%
\>[4]\AgdaSymbol{(}\AgdaFunction{var\#}\AgdaSpace{}%
\AgdaNumber{0}\AgdaSpace{}%
\AgdaOperator{\AgdaInductiveConstructor{·}}\AgdaSpace{}%
\AgdaSymbol{(}\AgdaFunction{var\#}\AgdaSpace{}%
\AgdaNumber{0}\AgdaSpace{}%
\AgdaOperator{\AgdaInductiveConstructor{·}}\AgdaSpace{}%
\AgdaInductiveConstructor{zero}\AgdaSymbol{))}\AgdaSpace{}%
\AgdaOperator{\AgdaFunction{[}}\AgdaSpace{}%
\AgdaOperator{\AgdaInductiveConstructor{ƛ}}\AgdaSpace{}%
\AgdaInductiveConstructor{delimit}\AgdaSpace{}%
\AgdaSymbol{(}\AgdaInductiveConstructor{suc}\AgdaSpace{}%
\AgdaSymbol{(}\AgdaFunction{var\#}\AgdaSpace{}%
\AgdaNumber{0}\AgdaSymbol{))}\AgdaSpace{}%
\AgdaOperator{\AgdaFunction{]}}\<%
\\
%
\>[2]\AgdaFunction{≡⟨⟩}\<%
\\
\>[2][@{}l@{\AgdaIndent{0}}]%
\>[4]\AgdaSymbol{(}\AgdaOperator{\AgdaInductiveConstructor{ƛ}}\AgdaSpace{}%
\AgdaInductiveConstructor{delimit}\AgdaSpace{}%
\AgdaSymbol{(}\AgdaInductiveConstructor{suc}\AgdaSpace{}%
\AgdaSymbol{(}\AgdaFunction{var\#}\AgdaSpace{}%
\AgdaNumber{0}\AgdaSymbol{)))}\AgdaSpace{}%
\AgdaOperator{\AgdaInductiveConstructor{·}}\AgdaSpace{}%
\AgdaSymbol{((}\AgdaOperator{\AgdaInductiveConstructor{ƛ}}\AgdaSpace{}%
\AgdaInductiveConstructor{delimit}\AgdaSpace{}%
\AgdaSymbol{(}\AgdaInductiveConstructor{suc}\AgdaSpace{}%
\AgdaSymbol{(}\AgdaFunction{var\#}\AgdaSpace{}%
\AgdaNumber{0}\AgdaSymbol{)))}\AgdaSpace{}%
\AgdaOperator{\AgdaInductiveConstructor{·}}\AgdaSpace{}%
\AgdaInductiveConstructor{zero}\AgdaSymbol{)}\<%
\\
%
\>[2]\AgdaFunction{⟶⟨}\AgdaSpace{}%
\AgdaInductiveConstructor{cong-·₂}\AgdaSpace{}%
\AgdaSymbol{(}\AgdaOperator{\AgdaInductiveConstructor{ƛ}}\AgdaSpace{}%
\AgdaInductiveConstructor{delimit}\AgdaSpace{}%
\AgdaSymbol{(}\AgdaInductiveConstructor{suc}\AgdaSpace{}%
\AgdaSymbol{(}\AgdaFunction{var\#}\AgdaSpace{}%
\AgdaNumber{0}\AgdaSymbol{)))}\AgdaSpace{}%
\AgdaSymbol{(}\AgdaInductiveConstructor{app}\AgdaSpace{}%
\AgdaInductiveConstructor{zero}\AgdaSymbol{)}\AgdaSpace{}%
\AgdaFunction{⟩}\<%
\\
\>[2][@{}l@{\AgdaIndent{0}}]%
\>[4]\AgdaSymbol{(}\AgdaOperator{\AgdaInductiveConstructor{ƛ}}\AgdaSpace{}%
\AgdaInductiveConstructor{delimit}\AgdaSpace{}%
\AgdaSymbol{(}\AgdaInductiveConstructor{suc}\AgdaSpace{}%
\AgdaSymbol{(}\AgdaFunction{var\#}\AgdaSpace{}%
\AgdaNumber{0}\AgdaSymbol{)))}\AgdaSpace{}%
\AgdaOperator{\AgdaInductiveConstructor{·}}\AgdaSpace{}%
\AgdaInductiveConstructor{delimit}\AgdaSpace{}%
\AgdaSymbol{(}\AgdaInductiveConstructor{suc}\AgdaSpace{}%
\AgdaInductiveConstructor{zero}\AgdaSymbol{)}\<%
\\
%
\>[2]\AgdaFunction{⟶⟨}\AgdaSpace{}%
\AgdaInductiveConstructor{cong-·₂}\AgdaSpace{}%
\AgdaSymbol{(}\AgdaOperator{\AgdaInductiveConstructor{ƛ}}\AgdaSpace{}%
\AgdaInductiveConstructor{delimit}\AgdaSpace{}%
\AgdaSymbol{(}\AgdaInductiveConstructor{suc}\AgdaSpace{}%
\AgdaSymbol{(}\AgdaFunction{var\#}\AgdaSpace{}%
\AgdaNumber{0}\AgdaSymbol{)))}\AgdaSpace{}%
\AgdaSymbol{(}\AgdaInductiveConstructor{delimit-value}\AgdaSpace{}%
\AgdaSymbol{(}\AgdaInductiveConstructor{suc}\AgdaSpace{}%
\AgdaInductiveConstructor{zero}\AgdaSymbol{))}\AgdaSpace{}%
\AgdaFunction{⟩}\<%
\\
\>[2][@{}l@{\AgdaIndent{0}}]%
\>[4]\AgdaSymbol{(}\AgdaOperator{\AgdaInductiveConstructor{ƛ}}\AgdaSpace{}%
\AgdaInductiveConstructor{delimit}\AgdaSpace{}%
\AgdaSymbol{(}\AgdaInductiveConstructor{suc}\AgdaSpace{}%
\AgdaSymbol{(}\AgdaFunction{var\#}\AgdaSpace{}%
\AgdaNumber{0}\AgdaSymbol{)))}\AgdaSpace{}%
\AgdaOperator{\AgdaInductiveConstructor{·}}\AgdaSpace{}%
\AgdaInductiveConstructor{suc}\AgdaSpace{}%
\AgdaInductiveConstructor{zero}\<%
\\
%
\>[2]\AgdaFunction{⟶⟨}\AgdaSpace{}%
\AgdaInductiveConstructor{app}\AgdaSpace{}%
\AgdaSymbol{(}\AgdaInductiveConstructor{suc}\AgdaSpace{}%
\AgdaInductiveConstructor{zero}\AgdaSymbol{)}\AgdaSpace{}%
\AgdaFunction{⟩}\<%
\\
\>[2][@{}l@{\AgdaIndent{0}}]%
\>[4]\AgdaInductiveConstructor{delimit}\AgdaSpace{}%
\AgdaSymbol{(}\AgdaInductiveConstructor{suc}\AgdaSpace{}%
\AgdaSymbol{(}\AgdaInductiveConstructor{suc}\AgdaSpace{}%
\AgdaInductiveConstructor{zero}\AgdaSymbol{))}\<%
\\
%
\>[2]\AgdaFunction{⟶⟨}\AgdaSpace{}%
\AgdaInductiveConstructor{delimit-value}\AgdaSpace{}%
\AgdaSymbol{(}\AgdaInductiveConstructor{suc}\AgdaSpace{}%
\AgdaSymbol{(}\AgdaInductiveConstructor{suc}\AgdaSpace{}%
\AgdaSymbol{(}\AgdaInductiveConstructor{zero}\AgdaSymbol{)))}\AgdaSpace{}%
\AgdaFunction{⟩}\<%
\\
\>[2][@{}l@{\AgdaIndent{0}}]%
\>[4]\AgdaInductiveConstructor{suc}\AgdaSpace{}%
\AgdaSymbol{(}\AgdaInductiveConstructor{suc}\AgdaSpace{}%
\AgdaInductiveConstructor{zero}\AgdaSymbol{)}\<%
\\
%
\>[2]\AgdaOperator{\AgdaFunction{∎}}\<%
\\
%
\>[2]\AgdaKeyword{where}\AgdaSpace{}%
\AgdaKeyword{open}\AgdaSpace{}%
\AgdaModule{⟶*-Reasoning}\<%
\\
%
\\[\AgdaEmptyExtraSkip]%
\>[0]\AgdaFunction{plus}\AgdaSpace{}%
\AgdaSymbol{:}\AgdaSpace{}%
\AgdaDatatype{Tm}\AgdaSpace{}%
\AgdaInductiveConstructor{∙}\AgdaSpace{}%
\AgdaInductiveConstructor{ι}\AgdaSpace{}%
\AgdaSymbol{(}\AgdaInductiveConstructor{`ℕ}\AgdaSpace{}%
\AgdaOperator{\AgdaInductiveConstructor{⇒}}\AgdaSpace{}%
\AgdaInductiveConstructor{`ℕ}\AgdaSpace{}%
\AgdaOperator{\AgdaInductiveConstructor{⇒}}\AgdaSpace{}%
\AgdaInductiveConstructor{`ℕ}\AgdaSpace{}%
\AgdaOperator{\AgdaInductiveConstructor{!}}\AgdaSpace{}%
\AgdaInductiveConstructor{ι}\AgdaSpace{}%
\AgdaOperator{\AgdaInductiveConstructor{!}}\AgdaSpace{}%
\AgdaInductiveConstructor{ι}\AgdaSymbol{)}\<%
\\
\>[0]\AgdaFunction{plus}\AgdaSpace{}%
\AgdaSymbol{=}\AgdaSpace{}%
\AgdaOperator{\AgdaInductiveConstructor{μ}}\AgdaSpace{}%
\AgdaOperator{\AgdaInductiveConstructor{ƛ}}\AgdaSpace{}%
\AgdaOperator{\AgdaInductiveConstructor{ƛ}}\AgdaSpace{}%
\AgdaInductiveConstructor{case}\AgdaSpace{}%
\AgdaSymbol{(}\AgdaFunction{var\#}\AgdaSpace{}%
\AgdaNumber{1}\AgdaSymbol{)}\AgdaSpace{}%
\AgdaSymbol{(}\AgdaFunction{var\#}\AgdaSpace{}%
\AgdaNumber{0}\AgdaSymbol{)}\AgdaSpace{}%
\AgdaSymbol{(}\AgdaInductiveConstructor{suc}\AgdaSpace{}%
\AgdaSymbol{(}\AgdaFunction{var\#}\AgdaSpace{}%
\AgdaNumber{3}\AgdaSpace{}%
\AgdaOperator{\AgdaInductiveConstructor{·}}\AgdaSpace{}%
\AgdaFunction{var\#}\AgdaSpace{}%
\AgdaNumber{0}\AgdaSpace{}%
\AgdaOperator{\AgdaInductiveConstructor{·}}\AgdaSpace{}%
\AgdaFunction{var\#}\AgdaSpace{}%
\AgdaNumber{1}\AgdaSymbol{))}\<%
\\
%
\\[\AgdaEmptyExtraSkip]%
\>[0]\AgdaFunction{\AgdaUnderscore{}}\AgdaSpace{}%
\AgdaSymbol{:}\AgdaSpace{}%
\AgdaFunction{plus}\AgdaSpace{}%
\AgdaOperator{\AgdaInductiveConstructor{·}}\AgdaSpace{}%
\AgdaInductiveConstructor{suc}\AgdaSpace{}%
\AgdaInductiveConstructor{zero}\AgdaSpace{}%
\AgdaOperator{\AgdaInductiveConstructor{·}}\AgdaSpace{}%
\AgdaInductiveConstructor{suc}\AgdaSpace{}%
\AgdaInductiveConstructor{zero}\AgdaSpace{}%
\AgdaOperator{\AgdaFunction{⟶*}}\AgdaSpace{}%
\AgdaInductiveConstructor{suc}\AgdaSpace{}%
\AgdaSymbol{(}\AgdaInductiveConstructor{suc}\AgdaSpace{}%
\AgdaInductiveConstructor{zero}\AgdaSymbol{)}\<%
\\
\>[0]\AgdaSymbol{\AgdaUnderscore{}}%
\>[1573I]\AgdaSymbol{=}\<%
\\
\>[.][@{}l@{}]\<[1573I]%
\>[2]\AgdaOperator{\AgdaFunction{begin}}\<%
\\
\>[2][@{}l@{\AgdaIndent{0}}]%
\>[4]\AgdaFunction{plus}\AgdaSpace{}%
\AgdaOperator{\AgdaInductiveConstructor{·}}\AgdaSpace{}%
\AgdaInductiveConstructor{suc}\AgdaSpace{}%
\AgdaInductiveConstructor{zero}\AgdaSpace{}%
\AgdaOperator{\AgdaInductiveConstructor{·}}\AgdaSpace{}%
\AgdaInductiveConstructor{suc}\AgdaSpace{}%
\AgdaInductiveConstructor{zero}\<%
\\
%
\>[2]\AgdaFunction{⟶⟨}\AgdaSpace{}%
\AgdaInductiveConstructor{cong-·₁}\AgdaSpace{}%
\AgdaSymbol{(}\AgdaInductiveConstructor{cong-·₁}\AgdaSpace{}%
\AgdaInductiveConstructor{unroll}\AgdaSymbol{)}\AgdaSpace{}%
\AgdaFunction{⟩}\<%
\\
\>[2][@{}l@{\AgdaIndent{0}}]%
\>[4]\AgdaSymbol{(}\AgdaOperator{\AgdaInductiveConstructor{ƛ}}\AgdaSpace{}%
\AgdaOperator{\AgdaInductiveConstructor{ƛ}}\AgdaSpace{}%
\AgdaInductiveConstructor{case}\AgdaSpace{}%
\AgdaSymbol{(}\AgdaFunction{var\#}\AgdaSpace{}%
\AgdaNumber{1}\AgdaSymbol{)}\AgdaSpace{}%
\AgdaSymbol{(}\AgdaFunction{var\#}\AgdaSpace{}%
\AgdaNumber{0}\AgdaSymbol{)}\AgdaSpace{}%
\AgdaSymbol{(}\AgdaInductiveConstructor{suc}\AgdaSpace{}%
\AgdaSymbol{((}\AgdaOperator{\AgdaInductiveConstructor{ƛ}}\AgdaSpace{}%
\AgdaSymbol{(}\AgdaOperator{\AgdaFunction{↑}}\AgdaSpace{}%
\AgdaOperator{\AgdaFunction{↑}}\AgdaSpace{}%
\AgdaOperator{\AgdaFunction{↑}}\AgdaSpace{}%
\AgdaOperator{\AgdaFunction{↑}}\AgdaSpace{}%
\AgdaFunction{plus}\AgdaSymbol{)}\AgdaSpace{}%
\AgdaOperator{\AgdaInductiveConstructor{·}}\AgdaSpace{}%
\AgdaFunction{var\#}\AgdaSpace{}%
\AgdaNumber{0}\AgdaSymbol{)}\AgdaSpace{}%
\AgdaOperator{\AgdaInductiveConstructor{·}}\AgdaSpace{}%
\AgdaFunction{var\#}\AgdaSpace{}%
\AgdaNumber{0}\AgdaSpace{}%
\AgdaOperator{\AgdaInductiveConstructor{·}}\AgdaSpace{}%
\AgdaFunction{var\#}\AgdaSpace{}%
\AgdaNumber{1}\AgdaSymbol{)))}\<%
\\
\>[4][@{}l@{\AgdaIndent{0}}]%
\>[6]\AgdaOperator{\AgdaInductiveConstructor{·}}\AgdaSpace{}%
\AgdaInductiveConstructor{suc}\AgdaSpace{}%
\AgdaInductiveConstructor{zero}\<%
\\
%
\>[6]\AgdaOperator{\AgdaInductiveConstructor{·}}\AgdaSpace{}%
\AgdaInductiveConstructor{suc}\AgdaSpace{}%
\AgdaInductiveConstructor{zero}\<%
\\
%
\>[2]\AgdaFunction{⟶⟨}\AgdaSpace{}%
\AgdaInductiveConstructor{cong-·₁}\AgdaSpace{}%
\AgdaSymbol{(}\AgdaInductiveConstructor{app}\AgdaSpace{}%
\AgdaSymbol{(}\AgdaInductiveConstructor{suc}\AgdaSpace{}%
\AgdaInductiveConstructor{zero}\AgdaSymbol{))}\AgdaSpace{}%
\AgdaFunction{⟩}\<%
\\
\>[2][@{}l@{\AgdaIndent{0}}]%
\>[4]\AgdaSymbol{(}\AgdaOperator{\AgdaInductiveConstructor{ƛ}}\AgdaSpace{}%
\AgdaInductiveConstructor{case}\AgdaSpace{}%
\AgdaSymbol{(}\AgdaInductiveConstructor{suc}\AgdaSpace{}%
\AgdaInductiveConstructor{zero}\AgdaSymbol{)}\AgdaSpace{}%
\AgdaSymbol{(}\AgdaFunction{var\#}\AgdaSpace{}%
\AgdaNumber{0}\AgdaSymbol{)}\AgdaSpace{}%
\AgdaSymbol{(}\AgdaInductiveConstructor{suc}\AgdaSpace{}%
\AgdaSymbol{((}\AgdaOperator{\AgdaInductiveConstructor{ƛ}}\AgdaSpace{}%
\AgdaSymbol{(}\AgdaOperator{\AgdaFunction{↑}}\AgdaSpace{}%
\AgdaOperator{\AgdaFunction{↑}}\AgdaSpace{}%
\AgdaOperator{\AgdaFunction{↑}}\AgdaSpace{}%
\AgdaFunction{plus}\AgdaSymbol{)}\AgdaSpace{}%
\AgdaOperator{\AgdaInductiveConstructor{·}}\AgdaSpace{}%
\AgdaFunction{var\#}\AgdaSpace{}%
\AgdaNumber{0}\AgdaSymbol{)}\AgdaSpace{}%
\AgdaOperator{\AgdaInductiveConstructor{·}}\AgdaSpace{}%
\AgdaFunction{var\#}\AgdaSpace{}%
\AgdaNumber{0}\AgdaSpace{}%
\AgdaOperator{\AgdaInductiveConstructor{·}}\AgdaSpace{}%
\AgdaFunction{var\#}\AgdaSpace{}%
\AgdaNumber{1}\AgdaSymbol{)))}\<%
\\
\>[4][@{}l@{\AgdaIndent{0}}]%
\>[6]\AgdaOperator{\AgdaInductiveConstructor{·}}\AgdaSpace{}%
\AgdaInductiveConstructor{suc}\AgdaSpace{}%
\AgdaInductiveConstructor{zero}\<%
\\
%
\>[2]\AgdaFunction{⟶⟨}\AgdaSpace{}%
\AgdaInductiveConstructor{app}\AgdaSpace{}%
\AgdaSymbol{(}\AgdaInductiveConstructor{suc}\AgdaSpace{}%
\AgdaInductiveConstructor{zero}\AgdaSymbol{)}\AgdaSpace{}%
\AgdaFunction{⟩}\<%
\\
\>[2][@{}l@{\AgdaIndent{0}}]%
\>[4]\AgdaInductiveConstructor{case}\AgdaSpace{}%
\AgdaSymbol{(}\AgdaInductiveConstructor{suc}\AgdaSpace{}%
\AgdaInductiveConstructor{zero}\AgdaSymbol{)}\AgdaSpace{}%
\AgdaSymbol{(}\AgdaInductiveConstructor{suc}\AgdaSpace{}%
\AgdaInductiveConstructor{zero}\AgdaSymbol{)}\AgdaSpace{}%
\AgdaSymbol{(}\AgdaInductiveConstructor{suc}\AgdaSpace{}%
\AgdaSymbol{((}\AgdaOperator{\AgdaInductiveConstructor{ƛ}}\AgdaSpace{}%
\AgdaOperator{\AgdaFunction{↑}}\AgdaSpace{}%
\AgdaOperator{\AgdaFunction{↑}}\AgdaSpace{}%
\AgdaFunction{plus}\AgdaSpace{}%
\AgdaOperator{\AgdaInductiveConstructor{·}}\AgdaSpace{}%
\AgdaFunction{var\#}\AgdaSpace{}%
\AgdaNumber{0}\AgdaSymbol{)}\AgdaSpace{}%
\AgdaOperator{\AgdaInductiveConstructor{·}}\AgdaSpace{}%
\AgdaFunction{var\#}\AgdaSpace{}%
\AgdaNumber{0}\AgdaSpace{}%
\AgdaOperator{\AgdaInductiveConstructor{·}}\AgdaSpace{}%
\AgdaInductiveConstructor{suc}\AgdaSpace{}%
\AgdaInductiveConstructor{zero}\AgdaSymbol{))}\<%
\\
%
\>[2]\AgdaFunction{⟶⟨}\AgdaSpace{}%
\AgdaInductiveConstructor{case-suc}\AgdaSpace{}%
\AgdaInductiveConstructor{zero}\AgdaSpace{}%
\AgdaFunction{⟩}\<%
\\
\>[2][@{}l@{\AgdaIndent{0}}]%
\>[4]\AgdaInductiveConstructor{suc}\AgdaSpace{}%
\AgdaSymbol{((}\AgdaOperator{\AgdaInductiveConstructor{ƛ}}\AgdaSpace{}%
\AgdaOperator{\AgdaFunction{↑}}\AgdaSpace{}%
\AgdaFunction{plus}\AgdaSpace{}%
\AgdaOperator{\AgdaInductiveConstructor{·}}\AgdaSpace{}%
\AgdaFunction{var\#}\AgdaSpace{}%
\AgdaNumber{0}\AgdaSymbol{)}\AgdaSpace{}%
\AgdaOperator{\AgdaInductiveConstructor{·}}\AgdaSpace{}%
\AgdaInductiveConstructor{zero}\AgdaSpace{}%
\AgdaOperator{\AgdaInductiveConstructor{·}}\AgdaSpace{}%
\AgdaInductiveConstructor{suc}\AgdaSpace{}%
\AgdaInductiveConstructor{zero}\AgdaSymbol{)}\<%
\\
%
\>[2]\AgdaFunction{⟶⟨}\AgdaSpace{}%
\AgdaInductiveConstructor{cong-suc}\AgdaSpace{}%
\AgdaSymbol{(}\AgdaInductiveConstructor{cong-·₁}\AgdaSpace{}%
\AgdaSymbol{(}\AgdaInductiveConstructor{app}\AgdaSpace{}%
\AgdaInductiveConstructor{zero}\AgdaSymbol{))}\AgdaSpace{}%
\AgdaFunction{⟩}\<%
\\
\>[2][@{}l@{\AgdaIndent{0}}]%
\>[4]\AgdaInductiveConstructor{suc}\AgdaSpace{}%
\AgdaSymbol{(}\AgdaFunction{plus}\AgdaSpace{}%
\AgdaOperator{\AgdaInductiveConstructor{·}}\AgdaSpace{}%
\AgdaInductiveConstructor{zero}\AgdaSpace{}%
\AgdaOperator{\AgdaInductiveConstructor{·}}\AgdaSpace{}%
\AgdaInductiveConstructor{suc}\AgdaSpace{}%
\AgdaInductiveConstructor{zero}\AgdaSymbol{)}\<%
\\
%
\>[2]\AgdaFunction{⟶⟨}\AgdaSpace{}%
\AgdaInductiveConstructor{cong-suc}\AgdaSpace{}%
\AgdaSymbol{(}\AgdaInductiveConstructor{cong-·₁}\AgdaSpace{}%
\AgdaSymbol{(}\AgdaInductiveConstructor{cong-·₁}\AgdaSpace{}%
\AgdaInductiveConstructor{unroll}\AgdaSymbol{))}\AgdaSpace{}%
\AgdaFunction{⟩}\<%
\\
\>[2][@{}l@{\AgdaIndent{0}}]%
\>[4]\AgdaInductiveConstructor{suc}\AgdaSpace{}%
\AgdaSymbol{(}\<%
\\
\>[4][@{}l@{\AgdaIndent{0}}]%
\>[6]\AgdaSymbol{(}\AgdaOperator{\AgdaInductiveConstructor{ƛ}}\AgdaSpace{}%
\AgdaOperator{\AgdaInductiveConstructor{ƛ}}\AgdaSpace{}%
\AgdaInductiveConstructor{case}\AgdaSpace{}%
\AgdaSymbol{(}\AgdaFunction{var\#}\AgdaSpace{}%
\AgdaNumber{1}\AgdaSymbol{)}\AgdaSpace{}%
\AgdaSymbol{(}\AgdaFunction{var\#}\AgdaSpace{}%
\AgdaNumber{0}\AgdaSymbol{)}\AgdaSpace{}%
\AgdaSymbol{(}\AgdaInductiveConstructor{suc}\AgdaSpace{}%
\AgdaSymbol{((}\AgdaOperator{\AgdaInductiveConstructor{ƛ}}\AgdaSpace{}%
\AgdaSymbol{(}\AgdaOperator{\AgdaFunction{↑}}\AgdaSpace{}%
\AgdaOperator{\AgdaFunction{↑}}\AgdaSpace{}%
\AgdaOperator{\AgdaFunction{↑}}\AgdaSpace{}%
\AgdaOperator{\AgdaFunction{↑}}\AgdaSpace{}%
\AgdaFunction{plus}\AgdaSymbol{)}\AgdaSpace{}%
\AgdaOperator{\AgdaInductiveConstructor{·}}\AgdaSpace{}%
\AgdaFunction{var\#}\AgdaSpace{}%
\AgdaNumber{0}\AgdaSymbol{)}\AgdaSpace{}%
\AgdaOperator{\AgdaInductiveConstructor{·}}\AgdaSpace{}%
\AgdaFunction{var\#}\AgdaSpace{}%
\AgdaNumber{0}\AgdaSpace{}%
\AgdaOperator{\AgdaInductiveConstructor{·}}\AgdaSpace{}%
\AgdaFunction{var\#}\AgdaSpace{}%
\AgdaNumber{1}\AgdaSymbol{)))}\<%
\\
%
\>[6]\AgdaOperator{\AgdaInductiveConstructor{·}}\AgdaSpace{}%
\AgdaInductiveConstructor{zero}\<%
\\
%
\>[6]\AgdaOperator{\AgdaInductiveConstructor{·}}\AgdaSpace{}%
\AgdaInductiveConstructor{suc}\AgdaSpace{}%
\AgdaInductiveConstructor{zero}\AgdaSymbol{)}\<%
\\
%
\>[2]\AgdaFunction{⟶⟨}\AgdaSpace{}%
\AgdaInductiveConstructor{cong-suc}\AgdaSpace{}%
\AgdaSymbol{(}\AgdaInductiveConstructor{cong-·₁}\AgdaSpace{}%
\AgdaSymbol{(}\AgdaInductiveConstructor{app}\AgdaSpace{}%
\AgdaInductiveConstructor{zero}\AgdaSymbol{))}\AgdaSpace{}%
\AgdaFunction{⟩}\<%
\\
\>[2][@{}l@{\AgdaIndent{0}}]%
\>[4]\AgdaInductiveConstructor{suc}\AgdaSpace{}%
\AgdaSymbol{((}\AgdaOperator{\AgdaInductiveConstructor{ƛ}}\AgdaSpace{}%
\AgdaInductiveConstructor{case}\AgdaSpace{}%
\AgdaInductiveConstructor{zero}\AgdaSpace{}%
\AgdaSymbol{(}\AgdaFunction{var\#}\AgdaSpace{}%
\AgdaNumber{0}\AgdaSymbol{)}\AgdaSpace{}%
\AgdaSymbol{(}\AgdaInductiveConstructor{suc}\AgdaSpace{}%
\AgdaSymbol{((}\AgdaOperator{\AgdaInductiveConstructor{ƛ}}\AgdaSpace{}%
\AgdaSymbol{(}\AgdaOperator{\AgdaFunction{↑}}\AgdaSpace{}%
\AgdaOperator{\AgdaFunction{↑}}\AgdaSpace{}%
\AgdaOperator{\AgdaFunction{↑}}\AgdaSpace{}%
\AgdaFunction{plus}\AgdaSymbol{)}\AgdaSpace{}%
\AgdaOperator{\AgdaInductiveConstructor{·}}\AgdaSpace{}%
\AgdaFunction{var\#}\AgdaSpace{}%
\AgdaNumber{0}\AgdaSymbol{)}\AgdaSpace{}%
\AgdaOperator{\AgdaInductiveConstructor{·}}\AgdaSpace{}%
\AgdaFunction{var\#}\AgdaSpace{}%
\AgdaNumber{0}\AgdaSpace{}%
\AgdaOperator{\AgdaInductiveConstructor{·}}\AgdaSpace{}%
\AgdaFunction{var\#}\AgdaSpace{}%
\AgdaNumber{1}\AgdaSymbol{)))}\AgdaSpace{}%
\AgdaOperator{\AgdaInductiveConstructor{·}}\AgdaSpace{}%
\AgdaInductiveConstructor{suc}\AgdaSpace{}%
\AgdaInductiveConstructor{zero}\AgdaSymbol{)}\<%
\\
%
\>[2]\AgdaFunction{⟶⟨}\AgdaSpace{}%
\AgdaInductiveConstructor{cong-suc}\AgdaSpace{}%
\AgdaSymbol{(}\AgdaInductiveConstructor{app}\AgdaSpace{}%
\AgdaSymbol{(}\AgdaInductiveConstructor{suc}\AgdaSpace{}%
\AgdaInductiveConstructor{zero}\AgdaSymbol{))}\AgdaSpace{}%
\AgdaFunction{⟩}\<%
\\
\>[2][@{}l@{\AgdaIndent{0}}]%
\>[4]\AgdaInductiveConstructor{suc}\AgdaSpace{}%
\AgdaSymbol{(}\AgdaInductiveConstructor{case}\AgdaSpace{}%
\AgdaInductiveConstructor{zero}\AgdaSpace{}%
\AgdaSymbol{(}\AgdaInductiveConstructor{suc}\AgdaSpace{}%
\AgdaInductiveConstructor{zero}\AgdaSymbol{)}\AgdaSpace{}%
\AgdaSymbol{(}\AgdaInductiveConstructor{suc}\AgdaSpace{}%
\AgdaSymbol{((}\AgdaOperator{\AgdaInductiveConstructor{ƛ}}\AgdaSpace{}%
\AgdaOperator{\AgdaFunction{↑}}\AgdaSpace{}%
\AgdaOperator{\AgdaFunction{↑}}\AgdaSpace{}%
\AgdaFunction{plus}\AgdaSpace{}%
\AgdaOperator{\AgdaInductiveConstructor{·}}\AgdaSpace{}%
\AgdaFunction{var\#}\AgdaSpace{}%
\AgdaNumber{0}\AgdaSymbol{)}\AgdaSpace{}%
\AgdaOperator{\AgdaInductiveConstructor{·}}\AgdaSpace{}%
\AgdaFunction{var\#}\AgdaSpace{}%
\AgdaNumber{0}\AgdaSpace{}%
\AgdaOperator{\AgdaInductiveConstructor{·}}\AgdaSpace{}%
\AgdaInductiveConstructor{suc}\AgdaSpace{}%
\AgdaInductiveConstructor{zero}\AgdaSymbol{)))}\<%
\\
%
\>[2]\AgdaFunction{⟶⟨}\AgdaSpace{}%
\AgdaInductiveConstructor{cong-suc}\AgdaSpace{}%
\AgdaInductiveConstructor{case-zero}\AgdaSpace{}%
\AgdaFunction{⟩}\<%
\\
\>[2][@{}l@{\AgdaIndent{0}}]%
\>[4]\AgdaInductiveConstructor{suc}\AgdaSpace{}%
\AgdaSymbol{(}\AgdaInductiveConstructor{suc}\AgdaSpace{}%
\AgdaInductiveConstructor{zero}\AgdaSymbol{)}\<%
\\
%
\>[2]\AgdaOperator{\AgdaFunction{∎}}\<%
\\
%
\>[2]\AgdaKeyword{where}\AgdaSpace{}%
\AgdaKeyword{open}\AgdaSpace{}%
\AgdaModule{⟶*-Reasoning}\<%
\end{code}

\subsection{Progress}

\begin{code}%
\>[0]\AgdaKeyword{data}\AgdaSpace{}%
\AgdaDatatype{Progress}\AgdaSpace{}%
\AgdaSymbol{:}\AgdaSpace{}%
\AgdaDatatype{Tm}\AgdaSpace{}%
\AgdaInductiveConstructor{∙}\AgdaSpace{}%
\AgdaGeneralizable{ε}\AgdaSpace{}%
\AgdaGeneralizable{α}\AgdaSpace{}%
\AgdaSymbol{→}\AgdaSpace{}%
\AgdaPrimitive{Set}\AgdaSpace{}%
\AgdaKeyword{where}\<%
\\
\>[0][@{}l@{\AgdaIndent{0}}]%
\>[2]\AgdaInductiveConstructor{done}\AgdaSpace{}%
\AgdaSymbol{:}\AgdaSpace{}%
\AgdaSymbol{\{}\AgdaBound{t}\AgdaSpace{}%
\AgdaSymbol{:}\AgdaSpace{}%
\AgdaDatatype{Tm}\AgdaSpace{}%
\AgdaInductiveConstructor{∙}\AgdaSpace{}%
\AgdaGeneralizable{ε}\AgdaSpace{}%
\AgdaGeneralizable{α}\AgdaSymbol{\}}\AgdaSpace{}%
\AgdaSymbol{→}\AgdaSpace{}%
\AgdaDatatype{Value}\AgdaSpace{}%
\AgdaBound{t}\AgdaSpace{}%
\AgdaSymbol{→}\AgdaSpace{}%
\AgdaDatatype{Progress}\AgdaSpace{}%
\AgdaBound{t}\<%
\\
%
\>[2]\AgdaInductiveConstructor{step}\AgdaSpace{}%
\AgdaSymbol{:}\AgdaSpace{}%
\AgdaSymbol{\{}\AgdaBound{t}\AgdaSpace{}%
\AgdaBound{t'}\AgdaSpace{}%
\AgdaSymbol{:}\AgdaSpace{}%
\AgdaDatatype{Tm}\AgdaSpace{}%
\AgdaInductiveConstructor{∙}\AgdaSpace{}%
\AgdaGeneralizable{ε}\AgdaSpace{}%
\AgdaGeneralizable{α}\AgdaSymbol{\}}\AgdaSpace{}%
\AgdaSymbol{→}\AgdaSpace{}%
\AgdaBound{t}\AgdaSpace{}%
\AgdaOperator{\AgdaDatatype{⟶}}\AgdaSpace{}%
\AgdaBound{t'}\AgdaSpace{}%
\AgdaSymbol{→}\AgdaSpace{}%
\AgdaDatatype{Progress}\AgdaSpace{}%
\AgdaBound{t}\<%
\\
%
\>[2]\AgdaInductiveConstructor{bare-grab}\AgdaSpace{}%
\AgdaSymbol{:}\AgdaSpace{}%
\AgdaSymbol{∀}\AgdaSpace{}%
\AgdaSymbol{\{}\AgdaBound{t}\AgdaSpace{}%
\AgdaSymbol{:}\AgdaSpace{}%
\AgdaDatatype{Tm}\AgdaSpace{}%
\AgdaInductiveConstructor{∙}\AgdaSpace{}%
\AgdaSymbol{(}\AgdaGeneralizable{ε}\AgdaSpace{}%
\AgdaOperator{\AgdaInductiveConstructor{,}}\AgdaSpace{}%
\AgdaGeneralizable{α}\AgdaSymbol{)}\AgdaSpace{}%
\AgdaGeneralizable{β}\AgdaSymbol{\}}\AgdaSpace{}%
\AgdaSymbol{(}\AgdaBound{c}\AgdaSpace{}%
\AgdaSymbol{:}\AgdaSpace{}%
\AgdaDatatype{PEC}\AgdaSpace{}%
\AgdaInductiveConstructor{∙}\AgdaSpace{}%
\AgdaSymbol{(}\AgdaGeneralizable{ε}\AgdaSpace{}%
\AgdaOperator{\AgdaInductiveConstructor{,}}\AgdaSpace{}%
\AgdaGeneralizable{α}\AgdaSymbol{)}\AgdaSpace{}%
\AgdaGeneralizable{γ}\AgdaSpace{}%
\AgdaGeneralizable{β}\AgdaSymbol{)}\AgdaSpace{}%
\AgdaSymbol{\{}\AgdaBound{u}\AgdaSymbol{\}}\<%
\\
\>[2][@{}l@{\AgdaIndent{0}}]%
\>[4]\AgdaSymbol{→}\AgdaSpace{}%
\AgdaBound{t}\AgdaSpace{}%
\AgdaOperator{\AgdaDatatype{≡}}\AgdaSpace{}%
\AgdaBound{c}\AgdaSpace{}%
\AgdaOperator{\AgdaFunction{⟨}}\AgdaSpace{}%
\AgdaInductiveConstructor{grab}\AgdaSpace{}%
\AgdaBound{u}\AgdaSpace{}%
\AgdaOperator{\AgdaFunction{⟩}}\<%
\\
%
\>[4]\AgdaSymbol{→}\AgdaSpace{}%
\AgdaDatatype{Progress}\AgdaSpace{}%
\AgdaBound{t}\<%
\\
%
\\[\AgdaEmptyExtraSkip]%
\>[0]\AgdaFunction{progress}\AgdaSpace{}%
\AgdaSymbol{:}\AgdaSpace{}%
\AgdaSymbol{(}\AgdaBound{t}\AgdaSpace{}%
\AgdaSymbol{:}\AgdaSpace{}%
\AgdaDatatype{Tm}\AgdaSpace{}%
\AgdaInductiveConstructor{∙}\AgdaSpace{}%
\AgdaGeneralizable{ε}\AgdaSpace{}%
\AgdaGeneralizable{α}\AgdaSymbol{)}\AgdaSpace{}%
\AgdaSymbol{→}\AgdaSpace{}%
\AgdaDatatype{Progress}\AgdaSpace{}%
\AgdaBound{t}\<%
\\
\>[0]\AgdaFunction{progress}\AgdaSpace{}%
\AgdaSymbol{(}\AgdaOperator{\AgdaInductiveConstructor{ƛ}}\AgdaSpace{}%
\AgdaBound{t}\AgdaSymbol{)}\AgdaSpace{}%
\AgdaSymbol{=}\AgdaSpace{}%
\AgdaInductiveConstructor{done}\AgdaSpace{}%
\AgdaSymbol{(}\AgdaOperator{\AgdaInductiveConstructor{ƛ}}\AgdaSpace{}%
\AgdaBound{t}\AgdaSymbol{)}\<%
\\
\>[0]\AgdaFunction{progress}\AgdaSpace{}%
\AgdaSymbol{(}\AgdaBound{t}\AgdaSpace{}%
\AgdaOperator{\AgdaInductiveConstructor{·}}\AgdaSpace{}%
\AgdaBound{u}\AgdaSymbol{)}\AgdaSpace{}%
\AgdaKeyword{with}\AgdaSpace{}%
\AgdaFunction{progress}\AgdaSpace{}%
\AgdaBound{t}\<%
\\
\>[0]\AgdaSymbol{...}\AgdaSpace{}%
\AgdaSymbol{|}\AgdaSpace{}%
\AgdaInductiveConstructor{step}\AgdaSpace{}%
\AgdaBound{t⟶t'}\AgdaSpace{}%
\AgdaSymbol{=}\AgdaSpace{}%
\AgdaInductiveConstructor{step}\AgdaSpace{}%
\AgdaSymbol{(}\AgdaInductiveConstructor{cong-·₁}\AgdaSpace{}%
\AgdaBound{t⟶t'}\AgdaSymbol{)}\<%
\\
\>[0]\AgdaSymbol{...}\AgdaSpace{}%
\AgdaSymbol{|}\AgdaSpace{}%
\AgdaInductiveConstructor{bare-grab}\AgdaSpace{}%
\AgdaBound{c}\AgdaSpace{}%
\AgdaBound{eq}\AgdaSpace{}%
\AgdaSymbol{=}\AgdaSpace{}%
\AgdaInductiveConstructor{bare-grab}\AgdaSpace{}%
\AgdaSymbol{(}\AgdaBound{c}\AgdaSpace{}%
\AgdaOperator{\AgdaInductiveConstructor{·₁}}\AgdaSpace{}%
\AgdaBound{u}\AgdaSymbol{)}\AgdaSpace{}%
\AgdaSymbol{(}\AgdaFunction{cong}\AgdaSpace{}%
\AgdaSymbol{(}\AgdaOperator{\AgdaInductiveConstructor{\AgdaUnderscore{}·}}\AgdaSpace{}%
\AgdaBound{u}\AgdaSymbol{)}\AgdaSpace{}%
\AgdaBound{eq}\AgdaSymbol{)}\<%
\\
\>[0]\AgdaSymbol{...}\AgdaSpace{}%
\AgdaSymbol{|}\AgdaSpace{}%
\AgdaInductiveConstructor{done}\AgdaSpace{}%
\AgdaSymbol{(}\AgdaOperator{\AgdaInductiveConstructor{ƛ}}\AgdaSpace{}%
\AgdaBound{t'}\AgdaSymbol{)}\AgdaSpace{}%
\AgdaKeyword{with}\AgdaSpace{}%
\AgdaFunction{progress}\AgdaSpace{}%
\AgdaBound{u}\<%
\\
\>[0]\AgdaSymbol{...}%
\>[6]\AgdaSymbol{|}\AgdaSpace{}%
\AgdaInductiveConstructor{step}\AgdaSpace{}%
\AgdaBound{u⟶u'}\AgdaSpace{}%
\AgdaSymbol{=}\AgdaSpace{}%
\AgdaInductiveConstructor{step}\AgdaSpace{}%
\AgdaSymbol{(}\AgdaInductiveConstructor{cong-·₂}\AgdaSpace{}%
\AgdaSymbol{(}\AgdaOperator{\AgdaInductiveConstructor{ƛ}}\AgdaSpace{}%
\AgdaBound{t'}\AgdaSymbol{)}\AgdaSpace{}%
\AgdaBound{u⟶u'}\AgdaSymbol{)}\<%
\\
\>[0]\AgdaSymbol{...}%
\>[6]\AgdaSymbol{|}\AgdaSpace{}%
\AgdaInductiveConstructor{done}\AgdaSpace{}%
\AgdaBound{vu}\AgdaSpace{}%
\AgdaSymbol{=}\AgdaSpace{}%
\AgdaInductiveConstructor{step}\AgdaSpace{}%
\AgdaSymbol{(}\AgdaInductiveConstructor{app}\AgdaSpace{}%
\AgdaBound{vu}\AgdaSymbol{)}\<%
\\
\>[0]\AgdaSymbol{...}%
\>[6]\AgdaSymbol{|}\AgdaSpace{}%
\AgdaInductiveConstructor{bare-grab}\AgdaSpace{}%
\AgdaBound{c}\AgdaSpace{}%
\AgdaBound{eq}\AgdaSpace{}%
\AgdaSymbol{=}\AgdaSpace{}%
\AgdaInductiveConstructor{bare-grab}\AgdaSpace{}%
\AgdaSymbol{((}\AgdaOperator{\AgdaInductiveConstructor{ƛ}}\AgdaSpace{}%
\AgdaBound{t'}\AgdaSymbol{)}\AgdaSpace{}%
\AgdaOperator{\AgdaInductiveConstructor{·₂}}\AgdaSpace{}%
\AgdaBound{c}\AgdaSymbol{)}\AgdaSpace{}%
\AgdaSymbol{(}\AgdaFunction{cong}\AgdaSpace{}%
\AgdaSymbol{((}\AgdaOperator{\AgdaInductiveConstructor{ƛ}}\AgdaSpace{}%
\AgdaBound{t'}\AgdaSymbol{)}\AgdaSpace{}%
\AgdaOperator{\AgdaInductiveConstructor{·\AgdaUnderscore{}}}\AgdaSymbol{)}\AgdaSpace{}%
\AgdaBound{eq}\AgdaSymbol{)}\<%
\\
\>[0]\AgdaFunction{progress}\AgdaSpace{}%
\AgdaSymbol{(}\AgdaInductiveConstructor{delimit}\AgdaSpace{}%
\AgdaBound{t}\AgdaSymbol{)}\AgdaSpace{}%
\AgdaKeyword{with}\AgdaSpace{}%
\AgdaFunction{progress}\AgdaSpace{}%
\AgdaBound{t}\<%
\\
\>[0]\AgdaSymbol{...}\AgdaSpace{}%
\AgdaSymbol{|}\AgdaSpace{}%
\AgdaInductiveConstructor{done}\AgdaSpace{}%
\AgdaBound{vt}\AgdaSpace{}%
\AgdaSymbol{=}\AgdaSpace{}%
\AgdaInductiveConstructor{step}\AgdaSpace{}%
\AgdaSymbol{(}\AgdaInductiveConstructor{delimit-value}\AgdaSpace{}%
\AgdaBound{vt}\AgdaSymbol{)}\<%
\\
\>[0]\AgdaSymbol{...}\AgdaSpace{}%
\AgdaSymbol{|}\AgdaSpace{}%
\AgdaInductiveConstructor{step}\AgdaSpace{}%
\AgdaBound{t⟶t'}\AgdaSpace{}%
\AgdaSymbol{=}\AgdaSpace{}%
\AgdaInductiveConstructor{step}\AgdaSpace{}%
\AgdaSymbol{(}\AgdaInductiveConstructor{cong-delimit}\AgdaSpace{}%
\AgdaBound{t⟶t'}\AgdaSymbol{)}\<%
\\
\>[0]\AgdaSymbol{...}\AgdaSpace{}%
\AgdaSymbol{|}\AgdaSpace{}%
\AgdaInductiveConstructor{bare-grab}\AgdaSpace{}%
\AgdaBound{c}\AgdaSpace{}%
\AgdaBound{eq}\AgdaSpace{}%
\AgdaSymbol{=}\AgdaSpace{}%
\AgdaInductiveConstructor{step}\AgdaSpace{}%
\AgdaSymbol{(}\AgdaInductiveConstructor{delimit-grab}\AgdaSpace{}%
\AgdaBound{c}\AgdaSpace{}%
\AgdaBound{eq}\AgdaSymbol{)}\<%
\\
\>[0]\AgdaFunction{progress}\AgdaSpace{}%
\AgdaSymbol{(}\AgdaInductiveConstructor{grab}\AgdaSpace{}%
\AgdaBound{t}\AgdaSymbol{)}\AgdaSpace{}%
\AgdaSymbol{=}\AgdaSpace{}%
\AgdaInductiveConstructor{bare-grab}\AgdaSpace{}%
\AgdaInductiveConstructor{⟨⟩}\AgdaSpace{}%
\AgdaInductiveConstructor{refl}\<%
\\
\>[0]\AgdaFunction{progress}\AgdaSpace{}%
\AgdaInductiveConstructor{zero}\AgdaSpace{}%
\AgdaSymbol{=}\AgdaSpace{}%
\AgdaInductiveConstructor{done}\AgdaSpace{}%
\AgdaInductiveConstructor{zero}\<%
\\
\>[0]\AgdaFunction{progress}\AgdaSpace{}%
\AgdaSymbol{(}\AgdaInductiveConstructor{suc}\AgdaSpace{}%
\AgdaBound{t}\AgdaSymbol{)}\AgdaSpace{}%
\AgdaKeyword{with}\AgdaSpace{}%
\AgdaFunction{progress}\AgdaSpace{}%
\AgdaBound{t}\<%
\\
\>[0]\AgdaSymbol{...}\AgdaSpace{}%
\AgdaSymbol{|}\AgdaSpace{}%
\AgdaInductiveConstructor{done}\AgdaSpace{}%
\AgdaBound{vt}\AgdaSpace{}%
\AgdaSymbol{=}\AgdaSpace{}%
\AgdaInductiveConstructor{done}\AgdaSpace{}%
\AgdaSymbol{(}\AgdaInductiveConstructor{suc}\AgdaSpace{}%
\AgdaBound{vt}\AgdaSymbol{)}\<%
\\
\>[0]\AgdaSymbol{...}\AgdaSpace{}%
\AgdaSymbol{|}\AgdaSpace{}%
\AgdaInductiveConstructor{step}\AgdaSpace{}%
\AgdaBound{t⟶t'}\AgdaSpace{}%
\AgdaSymbol{=}\AgdaSpace{}%
\AgdaInductiveConstructor{step}\AgdaSpace{}%
\AgdaSymbol{(}\AgdaInductiveConstructor{cong-suc}\AgdaSpace{}%
\AgdaBound{t⟶t'}\AgdaSymbol{)}\<%
\\
\>[0]\AgdaSymbol{...}\AgdaSpace{}%
\AgdaSymbol{|}\AgdaSpace{}%
\AgdaInductiveConstructor{bare-grab}\AgdaSpace{}%
\AgdaBound{c}\AgdaSpace{}%
\AgdaBound{eq}\AgdaSpace{}%
\AgdaSymbol{=}\AgdaSpace{}%
\AgdaInductiveConstructor{bare-grab}\AgdaSpace{}%
\AgdaSymbol{(}\AgdaInductiveConstructor{suc}\AgdaSpace{}%
\AgdaBound{c}\AgdaSymbol{)}\AgdaSpace{}%
\AgdaSymbol{(}\AgdaFunction{cong}\AgdaSpace{}%
\AgdaInductiveConstructor{suc}\AgdaSpace{}%
\AgdaBound{eq}\AgdaSymbol{)}\<%
\\
\>[0]\AgdaFunction{progress}\AgdaSpace{}%
\AgdaSymbol{(}\AgdaInductiveConstructor{case}\AgdaSpace{}%
\AgdaBound{n}\AgdaSpace{}%
\AgdaBound{z}\AgdaSpace{}%
\AgdaBound{s}\AgdaSymbol{)}\AgdaSpace{}%
\AgdaKeyword{with}\AgdaSpace{}%
\AgdaFunction{progress}\AgdaSpace{}%
\AgdaBound{n}\<%
\\
\>[0]\AgdaSymbol{...}\AgdaSpace{}%
\AgdaSymbol{|}\AgdaSpace{}%
\AgdaInductiveConstructor{done}\AgdaSpace{}%
\AgdaInductiveConstructor{zero}\AgdaSpace{}%
\AgdaSymbol{=}\AgdaSpace{}%
\AgdaInductiveConstructor{step}\AgdaSpace{}%
\AgdaInductiveConstructor{case-zero}\<%
\\
\>[0]\AgdaSymbol{...}\AgdaSpace{}%
\AgdaSymbol{|}\AgdaSpace{}%
\AgdaInductiveConstructor{done}\AgdaSpace{}%
\AgdaSymbol{(}\AgdaInductiveConstructor{suc}\AgdaSpace{}%
\AgdaBound{vn}\AgdaSymbol{)}\AgdaSpace{}%
\AgdaSymbol{=}\AgdaSpace{}%
\AgdaInductiveConstructor{step}\AgdaSpace{}%
\AgdaSymbol{(}\AgdaInductiveConstructor{case-suc}\AgdaSpace{}%
\AgdaBound{vn}\AgdaSymbol{)}\<%
\\
\>[0]\AgdaSymbol{...}\AgdaSpace{}%
\AgdaSymbol{|}\AgdaSpace{}%
\AgdaInductiveConstructor{step}\AgdaSpace{}%
\AgdaBound{n⟶n'}\AgdaSpace{}%
\AgdaSymbol{=}\AgdaSpace{}%
\AgdaInductiveConstructor{step}\AgdaSpace{}%
\AgdaSymbol{(}\AgdaInductiveConstructor{cong-case}\AgdaSpace{}%
\AgdaBound{n⟶n'}\AgdaSymbol{)}\<%
\\
\>[0]\AgdaSymbol{...}\AgdaSpace{}%
\AgdaSymbol{|}\AgdaSpace{}%
\AgdaInductiveConstructor{bare-grab}\AgdaSpace{}%
\AgdaBound{c}\AgdaSpace{}%
\AgdaBound{eq}\AgdaSpace{}%
\AgdaSymbol{=}\AgdaSpace{}%
\AgdaInductiveConstructor{bare-grab}\AgdaSpace{}%
\AgdaSymbol{(}\AgdaInductiveConstructor{case}\AgdaSpace{}%
\AgdaBound{c}\AgdaSpace{}%
\AgdaBound{z}\AgdaSpace{}%
\AgdaBound{s}\AgdaSymbol{)}\AgdaSpace{}%
\AgdaSymbol{(}\AgdaFunction{cong}\AgdaSpace{}%
\AgdaSymbol{(λ}\AgdaSpace{}%
\AgdaBound{m}\AgdaSpace{}%
\AgdaSymbol{→}\AgdaSpace{}%
\AgdaInductiveConstructor{case}\AgdaSpace{}%
\AgdaBound{m}\AgdaSpace{}%
\AgdaBound{z}\AgdaSpace{}%
\AgdaBound{s}\AgdaSymbol{)}\AgdaSpace{}%
\AgdaBound{eq}\AgdaSymbol{)}\<%
\\
\>[0]\AgdaFunction{progress}\AgdaSpace{}%
\AgdaSymbol{(}\AgdaOperator{\AgdaInductiveConstructor{μ}}\AgdaSpace{}%
\AgdaBound{t}\AgdaSymbol{)}\AgdaSpace{}%
\AgdaSymbol{=}\AgdaSpace{}%
\AgdaInductiveConstructor{step}\AgdaSpace{}%
\AgdaInductiveConstructor{unroll}\<%
\end{code}

\begin{code}[hide]%
\>[0]\AgdaFunction{V¬⟶}\AgdaSpace{}%
\AgdaSymbol{:}\AgdaSpace{}%
\AgdaSymbol{\{}\AgdaBound{v}\AgdaSpace{}%
\AgdaBound{v'}\AgdaSpace{}%
\AgdaSymbol{:}\AgdaSpace{}%
\AgdaDatatype{Tm}\AgdaSpace{}%
\AgdaGeneralizable{Γ}\AgdaSpace{}%
\AgdaGeneralizable{ε}\AgdaSpace{}%
\AgdaGeneralizable{α}\AgdaSymbol{\}}\AgdaSpace{}%
\AgdaSymbol{→}\AgdaSpace{}%
\AgdaDatatype{Value}\AgdaSpace{}%
\AgdaBound{v}\AgdaSpace{}%
\AgdaSymbol{→}\AgdaSpace{}%
\AgdaOperator{\AgdaFunction{¬}}\AgdaSpace{}%
\AgdaSymbol{(}\AgdaBound{v}\AgdaSpace{}%
\AgdaOperator{\AgdaDatatype{⟶}}\AgdaSpace{}%
\AgdaBound{v'}\AgdaSymbol{)}\<%
\\
\>[0]\AgdaFunction{V¬⟶}\AgdaSpace{}%
\AgdaSymbol{(}\AgdaInductiveConstructor{suc}\AgdaSpace{}%
\AgdaBound{v}\AgdaSymbol{)}\AgdaSpace{}%
\AgdaSymbol{(}\AgdaInductiveConstructor{cong-suc}\AgdaSpace{}%
\AgdaBound{v⟶v'}\AgdaSymbol{)}\AgdaSpace{}%
\AgdaSymbol{=}\AgdaSpace{}%
\AgdaFunction{V¬⟶}\AgdaSpace{}%
\AgdaBound{v}\AgdaSpace{}%
\AgdaBound{v⟶v'}\<%
\\
%
\\[\AgdaEmptyExtraSkip]%
\>[0]\AgdaFunction{V-unique}\AgdaSpace{}%
\AgdaSymbol{:}\AgdaSpace{}%
\AgdaSymbol{\{}\AgdaBound{t}\AgdaSpace{}%
\AgdaSymbol{:}\AgdaSpace{}%
\AgdaDatatype{Tm}\AgdaSpace{}%
\AgdaGeneralizable{Γ}\AgdaSpace{}%
\AgdaGeneralizable{ε}\AgdaSpace{}%
\AgdaGeneralizable{α}\AgdaSymbol{\}}\AgdaSpace{}%
\AgdaSymbol{(}\AgdaBound{v}\AgdaSpace{}%
\AgdaBound{v'}\AgdaSpace{}%
\AgdaSymbol{:}\AgdaSpace{}%
\AgdaDatatype{Value}\AgdaSpace{}%
\AgdaBound{t}\AgdaSymbol{)}\AgdaSpace{}%
\AgdaSymbol{→}\AgdaSpace{}%
\AgdaBound{v}\AgdaSpace{}%
\AgdaOperator{\AgdaDatatype{≡}}\AgdaSpace{}%
\AgdaBound{v'}\<%
\\
\>[0]\AgdaFunction{V-unique}\AgdaSpace{}%
\AgdaSymbol{(}\AgdaOperator{\AgdaInductiveConstructor{ƛ}}\AgdaSpace{}%
\AgdaBound{t}\AgdaSymbol{)}\AgdaSpace{}%
\AgdaSymbol{(}\AgdaOperator{\AgdaInductiveConstructor{ƛ}}\AgdaSpace{}%
\AgdaDottedPattern{\AgdaSymbol{.}}\AgdaDottedPattern{\AgdaBound{t}}\AgdaSymbol{)}\AgdaSpace{}%
\AgdaSymbol{=}\AgdaSpace{}%
\AgdaInductiveConstructor{refl}\<%
\\
\>[0]\AgdaFunction{V-unique}\AgdaSpace{}%
\AgdaInductiveConstructor{zero}\AgdaSpace{}%
\AgdaInductiveConstructor{zero}\AgdaSpace{}%
\AgdaSymbol{=}\AgdaSpace{}%
\AgdaInductiveConstructor{refl}\<%
\\
\>[0]\AgdaFunction{V-unique}\AgdaSpace{}%
\AgdaSymbol{(}\AgdaInductiveConstructor{suc}\AgdaSpace{}%
\AgdaBound{v}\AgdaSymbol{)}\AgdaSpace{}%
\AgdaSymbol{(}\AgdaInductiveConstructor{suc}\AgdaSpace{}%
\AgdaBound{v'}\AgdaSymbol{)}\AgdaSpace{}%
\AgdaSymbol{=}\AgdaSpace{}%
\AgdaFunction{cong}\AgdaSpace{}%
\AgdaInductiveConstructor{suc}\AgdaSpace{}%
\AgdaSymbol{(}\AgdaFunction{V-unique}\AgdaSpace{}%
\AgdaBound{v}\AgdaSpace{}%
\AgdaBound{v'}\AgdaSymbol{)}\<%
\end{code}

\begin{code}[hide]%
\>[0]\AgdaComment{--\ ⟶-deterministic\ :\ \{t\ s\ u\ :\ Tm\ Γ\ ε\ α\}\ →\ t\ ⟶\ s\ →\ t\ ⟶\ u\ →\ s\ ≡\ u}\<%
\\
\>[0]\AgdaComment{--\ ⟶-deterministic\ (app\ \{t\ =\ t\}\ v)\ (app\ v')\ rewrite\ V-unique\ v\ v'\ =\ refl}\<%
\\
\>[0]\AgdaComment{--\ ⟶-deterministic\ (delimit-value\ v)\ (delimit-value\ v')\ rewrite\ V-unique\ v\ v'\ =\ refl}\<%
\\
\>[0]\AgdaComment{--\ ⟶-deterministic\ (delimit-grab\ c\ eq)\ (delimit-grab\ c'\ eq')\ =\ \{!\ \ \ !\}}\<%
\\
\>[0]\AgdaComment{--\ ⟶-deterministic\ rec-zero\ rec-zero\ =\ refl}\<%
\\
\>[0]\AgdaComment{--\ ⟶-deterministic\ (rec-suc\ v)\ (rec-suc\ v')\ rewrite\ V-unique\ v\ v'\ =\ refl}\<%
\\
\>[0]\AgdaComment{--\ ⟶-deterministic\ (cong-suc\ t⟶s)\ (cong-suc\ t⟶u)\ rewrite\ ⟶-deterministic\ t⟶s\ t⟶u\ =\ refl}\<%
\\
\>[0]\AgdaComment{--\ ⟶-deterministic\ (cong-·₁\ t⟶s)\ (cong-·₁\ t⟶u)\ rewrite\ ⟶-deterministic\ t⟶s\ t⟶u\ =\ refl}\<%
\\
\>[0]\AgdaComment{--\ ⟶-deterministic\ (cong-·₂\ \AgdaUnderscore{}\ t⟶s)\ (cong-·₂\ \AgdaUnderscore{}\ t⟶u)\ rewrite\ ⟶-deterministic\ t⟶s\ t⟶u\ =\ refl}\<%
\\
\>[0]\AgdaComment{--\ ⟶-deterministic\ (cong-delimit\ t⟶s)\ (cong-delimit\ t⟶u)\ =\ cong\ delimit\ (⟶-deterministic\ t⟶s\ t⟶u)}\<%
\\
\>[0]\AgdaComment{--\ ⟶-deterministic\ (cong-rec\ t⟶s)\ (cong-rec\ t⟶u)\ rewrite\ ⟶-deterministic\ t⟶s\ t⟶u\ =\ refl}\<%
\\
\>[0]\AgdaComment{--\ ⟶-deterministic\ (app\ v)\ (cong-·₂\ \AgdaUnderscore{}\ t⟶u)\ =\ ⊥-elim\ (V¬⟶\ v\ t⟶u)}\<%
\\
\>[0]\AgdaComment{--\ ⟶-deterministic\ (cong-·₂\ \AgdaUnderscore{}\ t⟶s)\ (app\ v)\ =\ ⊥-elim\ (V¬⟶\ v\ t⟶s)}\<%
\\
\>[0]\AgdaComment{--\ ⟶-deterministic\ (delimit-value\ v)\ (cong-delimit\ t⟶u)\ =\ ⊥-elim\ (V¬⟶\ v\ t⟶u)}\<%
\\
\>[0]\AgdaComment{--\ ⟶-deterministic\ (cong-delimit\ t⟶s)\ (delimit-value\ v)\ =\ ⊥-elim\ (V¬⟶\ v\ t⟶s)}\<%
\\
\>[0]\AgdaComment{--\ ⟶-deterministic\ (delimit-value\ v)\ (delimit-grab\ c\ refl)\ =\ \{!\ \ \ !\}\ --\ ⊥-elim\ (V¬⟨grab⟩\ v\ c)}\<%
\\
\>[0]\AgdaComment{--\ ⟶-deterministic\ (delimit-grab\ c\ refl)\ (delimit-value\ v)\ =\ \{!\ \ \ !\}\ --\ ⊥-elim\ (V¬⟨grab⟩\ v\ c)}\<%
\\
\>[0]\AgdaComment{--\ ⟶-deterministic\ (delimit-grab\ c\ refl)\ (cong-delimit\ t⟶u)\ =\ \{!\ \ \ !\}\ --\ ⊥-elim\ (⟨grab⟩¬⟶\ c\ t⟶u)}\<%
\\
\>[0]\AgdaComment{--\ ⟶-deterministic\ (cong-delimit\ t⟶s)\ (delimit-grab\ c\ refl)\ =\ \{!\ \ \ !\}\ --\ ⊥-elim\ (⟨grab⟩¬⟶\ c\ t⟶s)}\<%
\\
\>[0]\AgdaComment{--\ ⟶-deterministic\ (rec-suc\ v)\ (cong-rec\ t⟶u)\ =\ ⊥-elim\ (V¬⟶\ (suc\ v)\ t⟶u)}\<%
\\
\>[0]\AgdaComment{--\ ⟶-deterministic\ (cong-rec\ t⟶s)\ (rec-suc\ v)\ =\ ⊥-elim\ (V¬⟶\ (suc\ v)\ t⟶s)}\<%
\\
\>[0]\AgdaComment{--\ ⟶-deterministic\ (cong-·₁\ t⟶s)\ (cong-·₂\ v\ t⟶u)\ =\ ⊥-elim\ (V¬⟶\ v\ t⟶s)}\<%
\\
\>[0]\AgdaComment{--\ ⟶-deterministic\ (cong-·₂\ v\ t⟶s)\ (cong-·₁\ t⟶u)\ =\ ⊥-elim\ (V¬⟶\ v\ t⟶u)}\<%
\end{code}


\section{send/run}

\section{Effect Handlers}

\bibliographystyle{plain}
\bibliography{refs.bib}

\end{document}
